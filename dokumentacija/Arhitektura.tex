\chapter{Arhitektura i dizajn sustava}
		
%		\textbf{\textit{dio 1. revizije}}\\
%
%		\textit{ Potrebno je opisati stil arhitekture te identificirati: podsustave, preslikavanje na radnu platformu, spremišta podataka, mrežne protokole, globalni upravljački tok i sklopovsko-programske zahtjeve. Po točkama razraditi i popratiti odgovarajućim skicama:}
%	\begin{itemize}
%		\item 	\textit{izbor arhitekture temeljem principa oblikovanja pokazanih na predavanjima (objasniti zašto ste baš odabrali takvu arhitekturu)}
%		\item 	\textit{organizaciju sustava s najviše razine apstrakcije (npr. klijent-poslužitelj, baza podataka, datotečni sustav, grafičko sučelje)}
%		\item 	\textit{organizaciju aplikacije (npr. slojevi frontend i backend, MVC arhitektura) }		
%	\end{itemize}
		
		Jedan od najvažnijih koraka pri razvoju sustava je izbor i oblikovanje arhitekture sustava. Kako će našu aplikaciju koristiti više različitih korisnika istovremeno (vlasnici automobila, autoservisi i administratori), te stoga što autoservisi koriste različite uređaje kojima će pristupati sustavu, odlučen je razvoj web aplikacije. Time će se sustavu moći pristupati sa svih uređaja (mobilni uređaji, tableti, desktop računala) bez obzira na platformu uz uvjet da na uređaju postoji odgovarajući web preglednik čime se želi i privući što veći broj vlasnika automobila kao korisnika aplikacije.
		
		Na slici \ref{fig:arhitektura1} prikazana je klijent-poslužitelj arhitektura na kojoj će se temeljiti sustav. Korisnici aplikacije putem svojih web preglednika (putem \textit{HTTP} ili \textit{HTTPS} protokola) pristupaju poslužitelju na kojem je pokrenuta web aplikacija koja obrađuje njihove zahtjeve i komunicira s bazom podataka (pokrenutoj na istom ili nekom drugom poslužitelju) putem \textit{JDBC} protokola.
		
		\begin{figure}[h]
			\centering
			\includegraphics[width=0.7\linewidth]{slike/server-client}
			\caption{Osnovna arhitektura cijelog sustava}
			\label{fig:arhitektura1}
		\end{figure}
	
		
		Kao tip arhitekture uzet će se objektno usmjerena arhitektura budući da najbolje odgovara razvoju složene višekorisničke aplikacije. 
		
		Pri razvoju će se pratiti Model-pogled-nadglednik obrazac (engl. \textit{MVC: Model-View-Controller}) prikazan na slici \ref{fig:mvc} koji dijeli sustav na tri komponente:
		\begin{itemize}
			\item Model - komponenta koja je zadužena za upravljanje podacima te uključuje sve razrede čiji se podaci obrađuju
			\item Pogled - komponenta koja služi za prikaz podataka (modela) korisniku
			\item Nadglednik - komponenta koja prima i obrađuju korisničke zahtjeve na način da dohvaća, provjerava i mijenja podatke u modelu te ih šalje pogledu za prikaz
		\end{itemize}
		
		\begin{figure}[h]
			\centering
			\includegraphics[width=0.7\linewidth]{slike/mvc}
			\caption{Oblikovni obrazac Model-pogled-nadglednik}
			\label{fig:mvc}
		\end{figure}
		
		Korištenjem Model-pogled-nadglednik oblikovnog obrasca smanjit će se ovisnost između korisničkog sučelja i ostatka sustava, a zbog podjele odgovornosti između komponenti i smanjene međuovisnosti olakšat će se paralelni razvoj, testiranje i moguće buduće promjene u sustavu.
		
		Arhitektura aplikacije se može podijeliti i na \textit{backend} i \textit{frontend} sloj. \textit{Backend} sloj izvršavat će se na poslužitelju (i biti ravijen po \textit{MVC} obrascu) te će rezultat obrade klijentskih zahtjeva biti grafički prikaz stranice u jeziku \textit{HTML} koji se šalje klijentu i predstavlja \textit{frontend} sloj.

		

				
		\section{Baza podataka}
			
%			\textbf{\textit{dio 1. revizije}}\\
%			
%		\textit{Potrebno je opisati koju vrstu i implementaciju baze podataka ste odabrali, glavne komponente od kojih se sastoji i slično.}

		Sustav tijekom svojeg rada sprema i dohvaća različite podatke, koji će biti pohranjeni u bazu podataka. U bazi podataka se ti podaci spremaju u relacije (tablice) s definiranim atributima. Baza treba osigurati konzistentnost podataka kroz definirana ograničenja te primarne i strane ključeve, mogućnost istovremenog pristupa podacima te izdržljivost i mogućnost oporavka podataka u slučaju kvara.
		
		Za preslikavanje modela razreda u relacijski model podataka koristi se biblioteka \text{Hibernate}. To je ORM (\textit{Object-Relational Mapping}) okvir koji automatski preslikava razrede i njihove veze u relacijski model te generira i provodi SQL upite koje sam obrađuje.
		
		U razvojnoj okolini se koristi \textit{H2} baza podataka koja se čuva u memoriji za vrijeme pokretanja aplikacije te omogućava lako testiranje i dodavanje podataka. Ona je odabrana jer za nju već postoji podrška u \textit{Spring} okviru koji se koristi.
		
		Sustav će u produkciji koristiti \textit{PostgreSQL} bazu podataka, koja je besplatna i otvorenog koda. Ona je također dobro podržana u \textit{Spring} okviru, ima različite mogućnosti replikacije podataka koje bi se kasnije mogle implementirati i podržava većinu operativnih sustava. Ona može biti pokrenuta na istom poslužitelju na kojem je pokrenut i sustav ili na nekom drugom, a sustav se na nju povezuje putem JDBC protokola.
		
		
		
			\subsection{Opis tablica}
			

				% \textit{Svaku tablicu je potrebno opisati po zadanom predlošku. Lijevo se nalazi točno ime varijable u bazi podataka, u sredini se nalazi tip podataka, a desno se nalazi opis varijable. Svjetlozelenom bojom označite primarni ključ. Svjetlo plavom označite strani ključ}
				
				\textbf{app\_user} Ovaj entitet sadrži sve podatke o registriranom korisniku aplikacije te predstavlja vlasnika automobila, servisera i administratora ovisno o sadržaju atributa \textit{dtype}. Atributi koje sadrži su: tip, jedinstveni identifikacijski broj, e-mail adresa, lozinka, korisničko ime, ime, prezime i OIB korisnika, te tip zaposlenika u autoservisu i jedinstveni identifikacijski broj autoservisa. Ovisno o tipu korisnika, neki atributi će imati vrijednost \textit{NULL} ako se na njih ne odnosi taj atribut. Entitet je u \textit{One-to-Many} vezi s entitetom vehicle preko atributa identifikacijaki broj korisnika. Također je u vezi \textit{Many-to-One} sa entitetom auto\_service preko jedinstvenog identifikatora autoservisa.
				
				
				\begin{longtabu} to \textwidth {|X[6, l]|X[6, l]|X[20, l]|}
					
					\hline \multicolumn{3}{|c|}{\textbf{app\_user (Korisnik)}}	 \\[3pt] \hline
					\endfirsthead
					
					\hline \multicolumn{3}{|c|}{\textbf{Korisnik}}	 \\[3pt] \hline
					\endhead
					
					\hline 
					\endlastfoot
					
					\cellcolor{LightGreen}
					dtype 				& VARCHAR	&  Tip korisnika: vlasnik vozila, zaposlenik autoservisa ili administrator.	 	\\ \hline
					userid				& BIGINT 	& Jedinstveni identifikacijski broj korisnika.  	\\ \hline 
					email 				& VARCHAR 	&  E-mail adresa korisnika. \\ \hline 
					first\_name 		& VARCHAR 	&  Ime korisnika. \\ \hline 
					last\_name 			& VARCHAR 	&  Prezime korisnika. \\ \hline 
					email 				& VARCHAR 	&  E-mail adresa korisnika. \\ \hline 
					password\_hash 		& VARCHAR	&  Lozinka korisnika.		\\ \hline 
					username 			& VARCHAR	&  Korisničko ime korisnika.		\\ \hline 
					oib 				& VARCHAR	&  OIB korisnika.		\\ \hline 
					employee\_type 		& VARCHAR	&  Tip zaposlenika u autoservisu - serviser ili vlasnik autoservisa.			\\ \hline 
					auto\_service\_id 	& BIGINT	&  Jedinstveni identifikacijski broj autoservisa.		\\ \hline 
					
					
				\end{longtabu}
			
			
			\textbf{vehicle} Ovo je entitet koji sadrži podatke o vozilu. Atributi su mu: jedinstveni identifikator, registarska oznaka, model i broj šasije vozila, te identifikacijski broj vlasnika vozila. Ovaj entitet je u vezi \textit{Many-to-One} s entitetom app\_user preko atributa identifikacijski broj vlasnika vozila. Također je u \textit{One-to-Many} vezi s entitetom repair\_order preko identifikacijakog broja vozila.
			
			
			
				\begin{longtabu} to \textwidth {|X[6, l]|X[6, l]|X[20, l]|}
					
					\hline \multicolumn{3}{|c|}{\textbf{vehicle (Vozilo)}}	 \\[3pt] \hline
					\endfirsthead
					
					\hline \multicolumn{3}{|c|}{\textbf{Vozilo}}	 \\[3pt] \hline
					\endhead
					
					\hline 
					\endlastfoot
					
					\cellcolor{LightGreen}
					vehicle\_id 			& BIGINT	&  	 Jedinstveni identifikator vozila.	\\ \hline
					license\_plate				& VARCHAR 	&  Registarska oznaka vozila. 	\\ \hline 
					vehicle\_model 				& VARCHAR 	&  Model vozila. \\ \hline 
					vin\_number 		& VARCHAR	&  Broj šasije vozila.		\\ \hline 
					owner\_user\_id 			& BIGINT	&  	Identifikacijski broj vlasnika vozila.	\\ \hline 
					
					
				\end{longtabu}
			
			\textbf{repair\_order} Ovo je entitet koji sadrži sve važne podatke o pojedinom radnom nalogu. Sadrži atribute: jedinstveni identifikacijski broj radnog naloga, identifikacijski broj autoservisa i vozila, te ukupna cijena i status radnog naloga (otvoren, zatvoren ili u obradi). Entitet je u vezi \textit{Many-to-One} s entitetom vehicle preko atributa identifikacijski broj vozila. Također je u \textit{Many-to-One} vezi s entitetom auto\_service preko identifikacijskog broja autoservisa. Ovaj entitet je generalizacija entiteta regular\_repair\_order i entiteta repairing\_repair\_order.
			
		
				\begin{longtabu} to \textwidth {|X[6, l]|X[6, l]|X[20, l]|}
					
					\hline \multicolumn{3}{|c|}{\textbf{repair\_order (Radni nalog)}}	 \\[3pt] \hline
					\endfirsthead
					
					\hline \multicolumn{3}{|c|}{\textbf{Radni nalog}}	 \\[3pt] \hline
					\endhead
					
					\hline 
					\endlastfoot
					
					\cellcolor{LightGreen}
					repair\_order\_id 				& BIGINT	&  	Jedinstveni identifikacijski broj radnog naloga. 	\\ \hline
					price				& DOUBLE PRECISION 	&   Ukupna cijena.	\\ \hline 
					service\_job\_ status 				& VARCHAR 	&  Status radnog naloga: otvoren, zatvoren ili u obradi.  \\ \hline 
					auto\_service\_id 		& BIGINT	&  	Identifikacijski broj autoservisa.	\\ \hline 
					vehicle\_id 			& BIGINT	&  	Identifikacijski broj vozila.	\\ \hline 
					
					
				\end{longtabu}
			\textbf{auto\_service} Ovaj entitet sadrži sve važne informacije o autoservisu. Atributi koje sadrži su: jedinstveni identifikator, ime, adresa i OIB autoservisa, te cijena redovitog servisa. Ovaj entitet povezan je vezom \textit{One-to-Many} preko atributa jedinstveni identifikator autoservisa s entitetima: repair\_order, app\_user, service\_labor i vehicle\_part.
			



				\begin{longtabu} to \textwidth {|X[6, l]|X[6, l]|X[20, l]|}
					
					\hline \multicolumn{3}{|c|}{\textbf{auto\_service (Autoservis)}}	 \\[3pt] \hline
					\endfirsthead
					
					\hline \multicolumn{3}{|c|}{\textbf{Autoservis}}	 \\[3pt] \hline
					\endhead
					
					\hline 
					\endlastfoot
					
					\cellcolor{LightGreen}
					auto\_service\_id 				& BIGINT	& Jedinstveni identifikator autoservisa.  	 	\\ \hline
					address				& VARCHAR 	&   Adresa autoservisa.	\\ \hline 
					oib 				& VARCHAR 	&   OIB autoservisa.\\ \hline 
					regular\_service \_price		& DOUBLE PRECISION	&  Cijena redovitog servisa.		\\ \hline 
					shop\_name 			& VARCHAR	&  		\\ \hline 
					
					
				\end{longtabu}
			
			\textbf{regular\_repair\_order} Ovaj entitet sadržava sve važne informacije o radnom nalogu koji se provodi za redoviti servis. Sadrži atribute: kilometražu vozila, primijećene kvarove, preporuku za izvanredni servis te ID radnog naloga kojem pripada. Ovaj entitet je specijalizacija entiteta repair\_order. 
			
				\begin{longtabu} to \textwidth {|X[6, l]|X[6, l]|X[20, l]|}
					
					\hline \multicolumn{3}{|c|}{\textbf{regular\_repair\_order (Radni nalog za redoviti servis)}}	 \\[3pt] \hline
					\endfirsthead
					
					\hline \multicolumn{3}{|c|}{\textbf{Radni nalog za redoviti servis}}	 \\[3pt] \hline
					\endhead
					
					\hline 
					\endlastfoot
					
					\cellcolor{LightGreen}
					kilometers 				& INTEGER	&  	Kilometraža vozila zabilježena na servisu.	\\ \hline
					observed\_ malfunctions				& VARCHAR 	&   Primijećeni kvarovi.	\\ \hline 
					repair\_ recommended 				& BOOLEAN 	&  Je li potreban izvanredni servis ili ne? \\ \hline 
					id		& BIGINT	&  	Identifikacijski broj radnog naloga.	\\ \hline 
					
					
				\end{longtabu}
			
			\noindent\textbf{repairing\_repair\_order} Ovaj entitet sadrži sve važne informacije za izvanredni servis vozila. Sadrži atribute: kvarove na vozilu i ID radnog naloga kojem pripada. Ovaj entitet je specijalizacija entiteta repair\_order. 
			
				\begin{longtabu} to \textwidth {|X[6, l]|X[6, l]|X[20, l]|}
					
					\hline \multicolumn{3}{|c|}{\textbf{repairing\_repair\_order (Radni nalog za izvanredni servis)}}	 \\[3pt] \hline
					\endfirsthead
					
					\hline \multicolumn{3}{|c|}{\textbf{repairing\_repair\_order}}	 \\[3pt] \hline
					\endhead
					
					\hline 
					\endlastfoot
					
					\cellcolor{LightGreen}
					malfunctions 				& VARCHAR	&  	 Kvarovi na vozilu.	\\ \hline
					id		& BIGINT	&  	Identifikacijski broj radnog naloga.	\\ \hline 
					
					
				\end{longtabu}
			
			\noindent\textbf{repair\_order\_item} Ovaj entitet sadrži sve važne informacije za pojedinu stavku na radnom nalogu. Atributi ovog entiteta su: identifikator stavke na radnom nalogu, njen naziv, cijena te ID izvanrednog radnog naloga kojem stavka pripada. Ovaj entitet je u vezi \textit{Many-to-One} sa entitetom repair\_order preko ID-a tog radnog naloga.
			
				\begin{longtabu} to \textwidth {|X[6, l]|X[6, l]|X[20, l]|}
					
					\hline \multicolumn{3}{|c|}{\textbf{repair\_order\_item (Stavka na radnom nalogu)}}	 \\[3pt] \hline
					\endfirsthead
					
					\hline \multicolumn{3}{|c|}{\textbf{repair\_order\_item}}	 \\[3pt] \hline
					\endhead
					
					\hline 
					\endlastfoot
					
					\cellcolor{LightGreen}
					item\_id 				& BIGINT	&  	Identifikator stavke. 	\\ \hline
					name				& VARCHAR 	&   Naziv stavke.	\\ \hline 
					price 				& DOUBLE PRECISION 	&   Cijena stavke. \\ \hline 
					repair\_order\_id		& BIGINT	&  	Identifikator izvanrednog radnog naloga kojem pripada popravak.	\\ \hline 
					
					
				\end{longtabu}
			
			\noindent\textbf{service\_labor} Ovaj entitet sadrži sve informacije važne za servisnu uslugu. Atributi ovog entiteta su: ID usluge, cijena usluge, naziv usluge i ID autoservisa koji pruža uslugu. Ovaj entitet je u vezi \textit{Many-to-One} sa entitetom auto\_service preko ID-a autoservisa.
			
				\begin{longtabu} to \textwidth {|X[6, l]|X[6, l]|X[20, l]|}
					
					\hline \multicolumn{3}{|c|}{\textbf{service\_labor (Servisna usluga)}}	 \\[3pt] \hline
					\endfirsthead
					
					\hline \multicolumn{3}{|c|}{\textbf{Servisna usluga}}	 \\[3pt] \hline
					\endhead
					
					\hline 
					\endlastfoot
					
					\cellcolor{LightGreen}
					service\_labor\_ id 				& BIGINT	&  	 Identifikator usluge.	\\ \hline
					price				& DOUBLE PRECISION 	&   Cijena usluge.	\\ \hline 
					service\_name 				& VARCHAR 	&  Naziv usluge.  \\ \hline 
					auto\_service\_id		& BIGINT	&  	Identifikator autoservisa koji pruža uslugu.	\\ \hline 
					
					
				\end{longtabu}
			
			\noindent\textbf{vehicle\_part} Ovaj entitet sadrži sve važne informacije za rezervni dio automobila. Ovaj entitet sadrži atribute: ID dijela, kilometraža za koju je dio predviđen, naziv dijela, njegova cijena te ID autoservisa koji ima rezervni dio. Ovaj entitet je u vezi \textit{Many-to-One} sa entitetom auto\_service preko ID-a tog autoservisa.
			
				\begin{longtabu} to \textwidth {|X[6, l]|X[6, l]|X[20, l]|}
					
					\hline \multicolumn{3}{|c|}{\textbf{vehicle\_part (Rezervni dio)}}	 \\[3pt] \hline
					\endfirsthead
					
					\hline \multicolumn{3}{|c|}{\textbf{vehicle\_part}}	 \\[3pt] \hline
					\endhead
					
					\hline 
					\endlastfoot
					
					\cellcolor{LightGreen}
					part\_id 				& BIGINT	&  	Identifikator rezervnog dijela. 	\\ \hline
					estimated\_ duration\_in\_km				& INTEGER 	&   Kilometraža za koju je predviđen rezervni dio.	\\ \hline 
					part\_name 				& VARCHAR 	& Naziv rezervnog dijela.  \\ \hline 
					price		& DOUBLE PRECISION	&  	Cijena rezervnog dijela.	\\ \hline 
					auto\_service\_id		& BIGINT	&  	Identifikator autoservisa koji ima rezervni dio.	\\ \hline
					
					
				\end{longtabu}
			
			\subsection{Dijagram baze podataka}
%				\textit{ U ovom potpoglavlju potrebno je umetnuti dijagram baze podataka. Primarni i strani ključevi moraju biti označeni, a tablice povezane. Bazu podataka je potrebno normalizirati. Podsjetite se kolegija "Baze podataka".}
				
				\begin{figure}[h]
					\centering
					\includegraphics[width=1.0\linewidth]{dijagrami/er-diagram}
					\caption{Relacijski model baze podataka}
					\label{fig:er-diagram}
				\end{figure}
			
			\eject
			
			
		\section{Dijagram razreda}
		
%			\textit{Potrebno je priložiti dijagram razreda s pripadajućim opisom. Zbog preglednosti je moguće dijagram razlomiti na više njih, ali moraju biti grupirani prema sličnim razinama apstrakcije i srodnim funkcionalnostima.}\\
%			
%			\textbf{\textit{dio 1. revizije}}\\
%			
%			\textit{Prilikom prve predaje projekta, potrebno je priložiti potpuno razrađen dijagram razreda vezan uz \textbf{generičku funkcionalnost} sustava. Ostale funkcionalnosti trebaju biti idejno razrađene u dijagramu sa sljedećim komponentama: nazivi razreda, nazivi metoda i vrste pristupa metodama (npr. javni, zaštićeni), nazivi atributa razreda, veze i odnosi između razreda.}\\
%			
%			\textbf{\textit{dio 2. revizije}}\\			
%			
%			\textit{Prilikom druge predaje projekta dijagram razreda i opisi moraju odgovarati stvarnom stanju implementacije}

		Razred \textit{AppUser} predstavlja registriranog korisnika koji nakon unosa svojih podataka može koristiti osnovne funkcionalnosti aplikacije. Razred \textit{Administrator} predstavlja administratora koji nakon registracije ima najveće ovlasti. Razred \textit{VehicleOwner} predstavlja registriranog vlasnika automobila. Razred \textit{Vehicle} predstavlja vozilo koje pripada nekom korisniku. Razred \textit{AutoService} označava jedan auto servis. Razred \textit{ServiceEmployee} predstavlja osobu zaposlenu u auto servisu, koja može biti vlasnik tog auto servisa ili serviser koji radi na automobilima. Razred \textit{RepairOrder} predstavlja radni nalog za servis. On ima dva podrazreda, razred \textit{RegularRepairOrder} koji označava radni nalog za redovni servis, te razred \textit{RepairingRepairOrder} koji označava radni nalog za izvanredni servis. Razred \textit{VehiclePart} predstavlja rezervni dio za automobil. Razred \textit{ServiceLabor} predstavlja uslugu koju serviser može obaviti na automobilu. Razred \textit{RepairOrderItem} označava dio automobila koji je ugrađen.
			
		\begin{figure}[h]
			\centering
			\includegraphics[width=1.0\linewidth]{dijagrami/class_diagram}
			\caption{Dijagram razreda modela}
			\label{fig:classdiagram}
		\end{figure}
			
			
			\eject
		
		\section{Dijagram stanja}
			
			
			\textbf{\textit{dio 2. revizije}}\\
			
			\textit{Potrebno je priložiti dijagram stanja i opisati ga. Dovoljan je jedan dijagram stanja koji prikazuje \textbf{značajan dio funkcionalnosti} sustava. Na primjer, stanja korisničkog sučelja i tijek korištenja neke ključne funkcionalnosti jesu značajan dio sustava, a registracija i prijava nisu. }
			
			
			\eject 
		
		\section{Dijagram aktivnosti}
			
			\textbf{\textit{dio 2. revizije}}\\
			
			 \textit{Potrebno je priložiti dijagram aktivnosti s pripadajućim opisom. Dijagram aktivnosti treba prikazivati značajan dio sustava.}
			
			\eject
		\section{Dijagram komponenti}
		
			\textbf{\textit{dio 2. revizije}}\\
		
			 \textit{Potrebno je priložiti dijagram komponenti s pripadajućim opisom. Dijagram komponenti treba prikazivati strukturu cijele aplikacije.}