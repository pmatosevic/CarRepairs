\chapter{Arhitektura i dizajn sustava}
		
%		\textbf{\textit{dio 1. revizije}}\\
%
%		\textit{ Potrebno je opisati stil arhitekture te identificirati: podsustave, preslikavanje na radnu platformu, spremišta podataka, mrežne protokole, globalni upravljački tok i sklopovsko-programske zahtjeve. Po točkama razraditi i popratiti odgovarajućim skicama:}
%	\begin{itemize}
%		\item 	\textit{izbor arhitekture temeljem principa oblikovanja pokazanih na predavanjima (objasniti zašto ste baš odabrali takvu arhitekturu)}
%		\item 	\textit{organizaciju sustava s najviše razine apstrakcije (npr. klijent-poslužitelj, baza podataka, datotečni sustav, grafičko sučelje)}
%		\item 	\textit{organizaciju aplikacije (npr. slojevi frontend i backend, MVC arhitektura) }		
%	\end{itemize}

	
	
		

		

				
		\section{Baza podataka}
			
%			\textbf{\textit{dio 1. revizije}}\\
%			
%		\textit{Potrebno je opisati koju vrstu i implementaciju baze podataka ste odabrali, glavne komponente od kojih se sastoji i slično.}

		Sustav tijekom svojeg rada sprema i dohvaća različite podatke, koji će biti pohranjeni u bazu podataka. U bazi podataka se ti podaci spremaju u relacije (tablice) s definiranim atributima. Baza treba osigurati konzistentnost podataka kroz definirana ograničenja te primarne i strane ključeve, mogućnost istovremenog pristupa podacima te izdržljivost i mogućnost oporavka podataka u slučaju kvara.
		
		Za preslikavanje modela razreda u relacijski model podataka koristi se biblioteka \text{Hibernate}. To je ORM (object-relational mapping) okvir koji automatski preslikava razrede i njihove veze u relacijski model te generira i provodi SQL upite koje sam obrađuje.
		
		U razvojnoj okolini se koristi \textit{H2} baza podataka koja se čuva u memoriji za vrijeme pokretanja aplikacije te omogućava lako testiranje i dodavanje podataka. Ona je odabrana jer za nju već postoji podrška u \textit{Spring} okviru koji se koristi.
		
		Sustav će u produkciji koristiti \textit{PostgreSQL} bazu podataka, koja je besplatna i otvorenog koda. Ona je također dobro podržana u \textit{Spring} okviru, ima različite mogućnosti replikacije podataka koje bi se kasnije mogle implementirati i podržava većinu operativnih sustava. Ona može biti pokrenuta na istom poslužitelju na kojem je pokrenut i sustav ili na nekom drugom, a sustav se na nju povezuje putem JDBC protokola.
		
		
		
			\subsection{Opis tablica}
			

				\textit{Svaku tablicu je potrebno opisati po zadanom predlošku. Lijevo se nalazi točno ime varijable u bazi podataka, u sredini se nalazi tip podataka, a desno se nalazi opis varijable. Svjetlozelenom bojom označite primarni ključ. Svjetlo plavom označite strani ključ}
				
				\begin{longtabu} to \textwidth {|X[6, l]|X[6, l]|X[20, l]|}
					
					\hline \multicolumn{3}{|c|}{\textbf{korisnik - ime tablice}}	 \\[3pt] \hline
					\endfirsthead
					
					\hline \multicolumn{3}{|c|}{\textbf{korisnik - ime tablice}}	 \\[3pt] \hline
					\endhead
					
					\hline 
					\endlastfoot
					
					\cellcolor{LightGreen}IDKorisnik & INT	&  	Lorem ipsum dolor sit amet, consectetur adipiscing elit, sed do eiusmod tempor incididunt ut labore et dolore magna aliqua. Ut enim ad minim veniam 	\\ \hline
					korisnickoIme	& VARCHAR &   	\\ \hline 
					email & VARCHAR &   \\ \hline 
					ime & VARCHAR	&  		\\ \hline 
					\cellcolor{LightBlue} primjer	& VARCHAR &   	\\ \hline 
					
					
				\end{longtabu}
			
			
			\subsection{Dijagram baze podataka}
%				\textit{ U ovom potpoglavlju potrebno je umetnuti dijagram baze podataka. Primarni i strani ključevi moraju biti označeni, a tablice povezane. Bazu podataka je potrebno normalizirati. Podsjetite se kolegija "Baze podataka".}
				
				\begin{figure}[h]
					\centering
					\includegraphics[width=1.0\linewidth]{dijagrami/er-diagram}
					\caption{Relacijski model baze podataka}
					\label{fig:er-diagram}
				\end{figure}
			
			\eject
			
			
		\section{Dijagram razreda}
		
			\textit{Potrebno je priložiti dijagram razreda s pripadajućim opisom. Zbog preglednosti je moguće dijagram razlomiti na više njih, ali moraju biti grupirani prema sličnim razinama apstrakcije i srodnim funkcionalnostima.}\\
			
			\textbf{\textit{dio 1. revizije}}\\
			
			\textit{Prilikom prve predaje projekta, potrebno je priložiti potpuno razrađen dijagram razreda vezan uz \textbf{generičku funkcionalnost} sustava. Ostale funkcionalnosti trebaju biti idejno razrađene u dijagramu sa sljedećim komponentama: nazivi razreda, nazivi metoda i vrste pristupa metodama (npr. javni, zaštićeni), nazivi atributa razreda, veze i odnosi između razreda.}\\
			
			\textbf{\textit{dio 2. revizije}}\\			
			
			\textit{Prilikom druge predaje projekta dijagram razreda i opisi moraju odgovarati stvarnom stanju implementacije}
			
			
			
			\eject
		
		\section{Dijagram stanja}
			
			
			\textbf{\textit{dio 2. revizije}}\\
			
			\textit{Potrebno je priložiti dijagram stanja i opisati ga. Dovoljan je jedan dijagram stanja koji prikazuje \textbf{značajan dio funkcionalnosti} sustava. Na primjer, stanja korisničkog sučelja i tijek korištenja neke ključne funkcionalnosti jesu značajan dio sustava, a registracija i prijava nisu. }
			
			
			\eject 
		
		\section{Dijagram aktivnosti}
			
			\textbf{\textit{dio 2. revizije}}\\
			
			 \textit{Potrebno je priložiti dijagram aktivnosti s pripadajućim opisom. Dijagram aktivnosti treba prikazivati značajan dio sustava.}
			
			\eject
		\section{Dijagram komponenti}
		
			\textbf{\textit{dio 2. revizije}}\\
		
			 \textit{Potrebno je priložiti dijagram komponenti s pripadajućim opisom. Dijagram komponenti treba prikazivati strukturu cijele aplikacije.}