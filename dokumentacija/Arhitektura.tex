\chapter{Arhitektura i dizajn sustava}
		
%		\textbf{\textit{dio 1. revizije}}\\
%
%		\textit{ Potrebno je opisati stil arhitekture te identificirati: podsustave, preslikavanje na radnu platformu, spremišta podataka, mrežne protokole, globalni upravljački tok i sklopovsko-programske zahtjeve. Po točkama razraditi i popratiti odgovarajućim skicama:}
%	\begin{itemize}
%		\item 	\textit{izbor arhitekture temeljem principa oblikovanja pokazanih na predavanjima (objasniti zašto ste baš odabrali takvu arhitekturu)}
%		\item 	\textit{organizaciju sustava s najviše razine apstrakcije (npr. klijent-poslužitelj, baza podataka, datotečni sustav, grafičko sučelje)}
%		\item 	\textit{organizaciju aplikacije (npr. slojevi frontend i backend, MVC arhitektura) }		
%	\end{itemize}
		
		Jedan od najvažnijih koraka pri razvoju sustava je izbor i oblikovanje arhitekture sustava. Kako će našu aplikaciju koristiti više različitih korisnika istovremeno (vlasnici automobila, autoservisi i administratori), te stoga što autoservisi koriste različite uređaje kojima će pristupati sustavu, odlučen je razvoj web aplikacije. Time će se sustavu moći pristupati sa svih uređaja (mobilni uređaji, tableti, desktop računala) bez obzira na platformu uz uvjet da na uređaju postoji odgovarajući web preglednik čime se želi i privući što veći broj vlasnika automobila kao korisnika aplikacije.
		
		Na slici \ref{fig:arhitektura1} prikazana je klijent-poslužitelj arhitektura na kojoj će se temeljiti sustav. Korisnici aplikacije putem svojih web preglednika (putem \textit{HTTP} ili \textit{HTTPS} protokola) pristupaju poslužitelju na kojem je pokrenuta web aplikacija koja obrađuje njihove zahtjeve i komunicira s bazom podataka (pokrenutoj na istom ili nekom drugom poslužitelju) putem \textit{JDBC} protokola.
		
		\begin{figure}[h]
			\centering
			\includegraphics[width=0.7\linewidth]{slike/server-client}
			\caption{Osnovna arhitektura cijelog sustava}
			\label{fig:arhitektura1}
		\end{figure}
	
		
		Kao tip arhitekture uzet će se objektno usmjerena arhitektura budući da najbolje odgovara razvoju složene višekorisničke aplikacije. 
		
		Pri razvoju će se pratiti Model-pogled-nadglednik obrazac (engl. \textit{MVC: Model-View-Controller}) prikazan na slici \ref{fig:mvc} koji dijeli sustav na tri komponente:
		\begin{packed_item}
			\item Model - komponenta koja je zadužena za upravljanje podacima te uključuje sve razrede čiji se podaci obrađuju
			\item Pogled - komponenta koja služi za prikaz podataka (modela) korisniku
			\item Nadglednik - komponenta koja prima i obrađuju korisničke zahtjeve na način da dohvaća, provjerava i mijenja podatke u modelu te ih šalje pogledu za prikaz
		\end{packed_item}
		
		\begin{figure}[h]
			\centering
			\includegraphics[width=0.7\linewidth]{slike/mvc}
			\caption{Oblikovni obrazac Model-pogled-nadglednik}
			\label{fig:mvc}
		\end{figure}
		
		Korištenjem Model-pogled-nadglednik oblikovnog obrasca smanjit će se ovisnost između korisničkog sučelja i ostatka sustava, a zbog podjele odgovornosti između komponenti i smanjene međuovisnosti olakšat će se paralelni razvoj, testiranje i moguće buduće promjene u sustavu.
		
		Arhitektura aplikacije se može podijeliti i na \textit{backend} i \textit{frontend} sloj. \textit{Backend} sloj izvršavat će se na poslužitelju (i biti ravijen po \textit{MVC} obrascu) te će rezultat obrade klijentskih zahtjeva biti grafički prikaz stranice. Takav prikaz u jezicima \textit{HTML}, \textit{CSS} i \textit{JavaScript} se šalje klijentu i predstavlja \textit{frontend} sloj.
		
		Prema pogledu korisnika (aktora) koji će koristiti sustav, on se može podijeliti na tri manja podsustava što je vidljivo na slici \ref{fig:arhitektura1}:
		\begin{packed_item}
			\item Aplikacija za vlasnike automobila
			\item Aplikacija za autoservise
			\item Aplikacija za administraciju 
		\end{packed_item}
	
		Sustav će biti implementiran u programskom jeziku \textit{Java} korištenjem \textit{Spring} okvira uz potrebna proširenja. To je okvir koji nudi dio poslužiteljske implementacije web aplikacije slijedeći obrazac Model-pogled-nadglednik. Uz to, slijedi princip inverzije ovisnosti koji će olakšati kasnije nadogradnje i promjene smanjivanjem ovisnosti te nudi gotovu implementaciju različitih protokola i automatsku postavljanje autentikacije. Time će se ubrzati razvoj, a dodatno će se uz njega koristiti i okvir za pristup podacima \textit{Spring JPA} te okvir \textit{Thymeleaf} namijenjen generiranju \textit{HTML} koda iz definiranih predložaka.
		
		\eject

		

				
		\section{Baza podataka}
			
%			\textbf{\textit{dio 1. revizije}}\\
%			
%		\textit{Potrebno je opisati koju vrstu i implementaciju baze podataka ste odabrali, glavne komponente od kojih se sastoji i slično.}

		Sustav tijekom svojeg rada sprema i dohvaća različite podatke, koji će biti pohranjeni u bazu podataka. U bazi podataka se ti podaci spremaju u relacije (tablice) s definiranim atributima. Baza treba osigurati konzistentnost podataka kroz definirana ograničenja te primarne i strane ključeve, mogućnost istovremenog pristupa podacima te izdržljivost i mogućnost oporavka podataka u slučaju kvara.
		
		Za preslikavanje modela razreda u relacijski model podataka koristi se biblioteka \text{Hibernate} unutar okvira \textit{Spring JPA}. To je ORM (\textit{Object-Relational Mapping}) okvir koji automatski preslikava razrede i njihove veze u relacijski model te generira i provodi SQL upite koje sam obrađuje.
		
		U razvojnoj okolini se koristi \textit{H2} baza podataka koja se čuva u memoriji za vrijeme pokretanja aplikacije te omogućava lako testiranje i dodavanje podataka. Ona je odabrana jer za nju već postoji podrška u \textit{Spring} okviru koji se koristi.
		
		Sustav će u produkciji koristiti \textit{PostgreSQL} bazu podataka, koja je besplatna i otvorenog koda. Ona je također dobro podržana u \textit{Spring} okviru, ima različite mogućnosti replikacije podataka koje bi se kasnije mogle implementirati i podržava većinu operativnih sustava. Ona može biti pokrenuta na istom poslužitelju na kojem je pokrenut i sustav ili na nekom drugom, a sustav se na nju povezuje putem JDBC protokola.
		
		
		
			\subsection{Opis tablica}
			

				% \textit{Svaku tablicu je potrebno opisati po zadanom predlošku. Lijevo se nalazi točno ime varijable u bazi podataka, u sredini se nalazi tip podataka, a desno se nalazi opis varijable. Svjetlozelenom bojom označite primarni ključ. Svjetlo plavom označite strani ključ}
				
				{\small \textbf{Napomena} Atributi koji su \textbf{podebljani} označavaju primarne ključeve, a oni koji su \underline{podcrtani} označavaju strane.}
				
				\noindent\textbf{app\_user} Ovaj entitet sadrži sve podatke o registriranom korisniku aplikacije te predstavlja vlasnika automobila, servisera i administratora ovisno o sadržaju atributa \textit{dtype}. Atributi koje sadrži su: tip, jedinstveni identifikacijski broj, e-mail adresa, lozinka, korisničko ime, ime, prezime i OIB korisnika, te tip zaposlenika u autoservisu i jedinstveni identifikacijski broj autoservisa. Ovisno o tipu korisnika, neki atributi će imati vrijednost \textit{NULL} ako se na njih ne odnosi taj atribut. Entitet je u \textit{One-to-Many} vezi s entitetom vehicle preko atributa identifikacijaki broj korisnika. Također je u vezi \textit{Many-to-One} sa entitetom auto\_service preko jedinstvenog identifikatora autoservisa.
				
				
				\begin{longtabu} to \textwidth {|X[6, l]|X[6, l]|X[20, l]|}
					
					\hline \multicolumn{3}{|c|}{\textbf{app\_user (Korisnik)}}	 \\[3pt] \hline
					\endfirsthead
					
					\hline \multicolumn{3}{|c|}{\textbf{app\_user}}	 \\[3pt] \hline
					\endhead
					
					\hline 
					\endlastfoot
					
					\cellcolor{LightGreen}
					dtype 				& VARCHAR	&  Tip korisnika: vlasnik vozila, zaposlenik autoservisa ili administrator.	 	\\ \hline
					\textbf{user\_id}				& BIGINT 	& Jedinstveni identifikacijski broj korisnika.  	\\ \hline 
					email 				& VARCHAR 	&  E-mail adresa korisnika. \\ \hline 
					first\_name 		& VARCHAR 	&  Ime korisnika. \\ \hline 
					last\_name 			& VARCHAR 	&  Prezime korisnika. \\ \hline 
					email 				& VARCHAR 	&  E-mail adresa korisnika. \\ \hline 
					password\_hash 		& VARCHAR	&  Lozinka korisnika.		\\ \hline 
					username 			& VARCHAR	&  Korisničko ime korisnika.		\\ \hline 
					oib 				& VARCHAR	&  OIB korisnika.		\\ \hline 
					employee\_type 		& VARCHAR	&  Tip zaposlenika u autoservisu - serviser ili vlasnik autoservisa.			\\ \hline 
					\underline{auto\_service\_id} 	& BIGINT	&  Jedinstveni identifikacijski broj autoservisa.		\\ \hline 
					
					
				\end{longtabu}
			
			
			\noindent\textbf{vehicle} Ovo je entitet koji sadrži podatke o vozilu. Atributi su mu: jedinstveni identifikator, registarska oznaka, model i broj šasije vozila, te identifikacijski broj vlasnika vozila. Ovaj entitet je u vezi \textit{Many-to-One} s entitetom app\_user preko atributa identifikacijski broj vlasnika vozila. Također je u \textit{One-to-Many} vezi s entitetom repair\_order preko identifikacijakog broja vozila.
			
			
			
				\begin{longtabu} to \textwidth {|X[6, l]|X[6, l]|X[20, l]|}
					
					\hline \multicolumn{3}{|c|}{\textbf{vehicle (Vozilo)}}	 \\[3pt] \hline
					\endfirsthead
					
					\hline \multicolumn{3}{|c|}{\textbf{vehicle}}	 \\[3pt] \hline
					\endhead
					
					\hline 
					\endlastfoot
					
					\cellcolor{LightGreen}
					\textbf{vehicle\_id} 			& BIGINT	&  	 Jedinstveni identifikator vozila.	\\ \hline
					license\_plate				& VARCHAR 	&  Registarska oznaka vozila. 	\\ \hline 
					vehicle\_model 				& VARCHAR 	&  Model vozila. \\ \hline 
					vin\_number 		& VARCHAR	&  Broj šasije vozila.		\\ \hline 
					\underline{owner\_user\_id} 			& BIGINT	&  	Identifikacijski broj vlasnika vozila.	\\ \hline 
					
					
				\end{longtabu}
			
			\noindent\textbf{repair\_order} Ovo je entitet koji sadrži sve važne podatke o pojedinom radnom nalogu. Sadrži atribute: jedinstveni identifikacijski broj radnog naloga, identifikacijski broj autoservisa i vozila, te ukupna cijena i status radnog naloga (otvoren, zatvoren ili u obradi). Entitet je u vezi \textit{Many-to-One} s entitetom vehicle preko atributa identifikacijski broj vozila. Također je u \textit{Many-to-One} vezi s entitetom auto\_service preko identifikacijskog broja autoservisa. Ovaj entitet je generalizacija entiteta regular\_repair\_order i entiteta repairing\_repair\_order.
			
		
				\begin{longtabu} to \textwidth {|X[6, l]|X[6, l]|X[20, l]|}
					
					\hline \multicolumn{3}{|c|}{\textbf{repair\_order (Radni nalog)}}	 \\[3pt] \hline
					\endfirsthead
					
					\hline \multicolumn{3}{|c|}{\textbf{repair\_order}}	 \\[3pt] \hline
					\endhead
					
					\hline 
					\endlastfoot
					
					\cellcolor{LightGreen}
					\textbf{repair\_order\_id} 				& BIGINT	&  	Jedinstveni identifikacijski broj radnog naloga. 	\\ \hline
					price				& DOUBLE PRECISION 	&   Ukupna cijena.	\\ \hline 
					service\_job\_ status 				& VARCHAR 	&  Status radnog naloga: otvoren, zatvoren ili u obradi.  \\ \hline 
					\underline{auto\_service\_id} 		& BIGINT	&  	Identifikacijski broj autoservisa.	\\ \hline 
					\underline{vehicle\_id} 			& BIGINT	&  	Identifikacijski broj vozila.	\\ \hline 
					
					
				\end{longtabu}
			\noindent\textbf{auto\_service} Ovaj entitet sadrži sve važne informacije o autoservisu. Atributi koje sadrži su: jedinstveni identifikator, ime, adresa i OIB autoservisa, te cijena redovitog servisa. Ovaj entitet povezan je vezom \textit{One-to-Many} preko atributa jedinstveni identifikator autoservisa s entitetima: repair\_order, app\_user, service\_labor i vehicle\_part.
			



				\begin{longtabu} to \textwidth {|X[6, l]|X[6, l]|X[20, l]|}
					
					\hline \multicolumn{3}{|c|}{\textbf{auto\_service (Autoservis)}}	 \\[3pt] \hline
					\endfirsthead
					
					\hline \multicolumn{3}{|c|}{\textbf{auto\_service}}	 \\[3pt] \hline
					\endhead
					
					\hline 
					\endlastfoot
					
					\cellcolor{LightGreen}
					\textbf{auto\_service\_id} 				& BIGINT	& Jedinstveni identifikator autoservisa.  	 	\\ \hline
					address				& VARCHAR 	&   Adresa autoservisa.	\\ \hline 
					oib 				& VARCHAR 	&   OIB autoservisa.\\ \hline 
					regular\_service \_price		& DOUBLE PRECISION	&  Cijena redovitog servisa.		\\ \hline 
					shop\_name 			& VARCHAR	& Naziv autoservisa. 		\\ \hline 
					
					
				\end{longtabu}
			
			\noindent\textbf{regular\_repair\_order} Ovaj entitet sadržava sve važne informacije o radnom nalogu koji se provodi za redoviti servis. Sadrži atribute: kilometražu vozila, primijećene kvarove, preporuku za izvanredni servis te ID radnog naloga kojem pripada. Ovaj entitet je specijalizacija entiteta repair\_order. 
			
				\begin{longtabu} to \textwidth {|X[6, l]|X[6, l]|X[20, l]|}
					
					\hline \multicolumn{3}{|c|}{\textbf{regular\_repair\_order (Radni nalog za redoviti servis)}}	 \\[3pt] \hline
					\endfirsthead
					
					\hline \multicolumn{3}{|c|}{\textbf{Radni nalog za redoviti servis}}	 \\[3pt] \hline
					\endhead
					
					\hline 
					\endlastfoot
					
					\cellcolor{LightGreen}
					kilometers 				& INTEGER	&  	Kilometraža vozila zabilježena na servisu.	\\ \hline
					observed\_ malfunctions				& VARCHAR 	&   Primijećeni kvarovi.	\\ \hline 
					repair\_ recommended 				& BOOLEAN 	&  Je li potreban izvanredni servis ili ne? \\ \hline 
					\textbf{\underline{id}}		& BIGINT	&  	Identifikacijski broj radnog naloga.	\\ \hline 
					
					
				\end{longtabu}
			
			\noindent\textbf{repairing\_repair\_order} Ovaj entitet sadrži sve važne informacije za izvanredni servis vozila. Sadrži atribute: kvarove na vozilu i ID radnog naloga kojem pripada. Ovaj entitet je specijalizacija entiteta repair\_order. 
			
				\begin{longtabu} to \textwidth {|X[6, l]|X[6, l]|X[20, l]|}
					
					\hline \multicolumn{3}{|c|}{\textbf{repairing\_repair\_order (Radni nalog za izvanredni servis)}}	 \\[3pt] \hline
					\endfirsthead
					
					\hline \multicolumn{3}{|c|}{\textbf{repairing\_repair\_order}}	 \\[3pt] \hline
					\endhead
					
					\hline 
					\endlastfoot
					
					\cellcolor{LightGreen}
					malfunctions 				& VARCHAR	&  	 Kvarovi na vozilu.	\\ \hline
					\textbf{\underline{id}}		& BIGINT	&  	Identifikacijski broj radnog naloga.	\\ \hline 
					
					
				\end{longtabu}
			
			\noindent\textbf{repair\_order\_item} Ovaj entitet sadrži sve važne informacije za pojedinu stavku na radnom nalogu. Atributi ovog entiteta su: identifikator stavke na radnom nalogu, njen naziv, cijena te ID izvanrednog radnog naloga kojem stavka pripada. Ovaj entitet je u vezi \textit{Many-to-One} sa entitetom repair\_order preko ID-a tog radnog naloga.
			
				\begin{longtabu} to \textwidth {|X[6, l]|X[6, l]|X[20, l]|}
					
					\hline \multicolumn{3}{|c|}{\textbf{repair\_order\_item (Stavka na radnom nalogu)}}	 \\[3pt] \hline
					\endfirsthead
					
					\hline \multicolumn{3}{|c|}{\textbf{repair\_order\_item}}	 \\[3pt] \hline
					\endhead
					
					\hline 
					\endlastfoot
					
					\cellcolor{LightGreen}
					\textbf{item\_id} 				& BIGINT	&  	Identifikator stavke. 	\\ \hline
					name				& VARCHAR 	&   Naziv stavke.	\\ \hline 
					price 				& DOUBLE PRECISION 	&   Cijena stavke. \\ \hline 
					\underline{repair\_order\_id}		& BIGINT	&  	Identifikator izvanrednog radnog naloga kojem pripada popravak.	\\ \hline 
					
					
				\end{longtabu}
			
			\noindent\textbf{service\_labor} Ovaj entitet sadrži sve informacije važne za servisnu uslugu. Atributi ovog entiteta su: ID usluge, cijena usluge, naziv usluge i ID autoservisa koji pruža uslugu. Ovaj entitet je u vezi \textit{Many-to-One} sa entitetom auto\_service preko ID-a autoservisa.
			
				\begin{longtabu} to \textwidth {|X[6, l]|X[6, l]|X[20, l]|}
					
					\hline \multicolumn{3}{|c|}{\textbf{service\_labor (Servisna usluga)}}	 \\[3pt] \hline
					\endfirsthead
					
					\hline \multicolumn{3}{|c|}{\textbf{service\_labor}}	 \\[3pt] \hline
					\endhead
					
					\hline 
					\endlastfoot
					
					\cellcolor{LightGreen}
					\textbf{service\_labor\_ id} 				& BIGINT	&  	 Identifikator usluge.	\\ \hline
					price				& DOUBLE PRECISION 	&   Cijena usluge.	\\ \hline 
					service\_name 				& VARCHAR 	&  Naziv usluge.  \\ \hline 
					\underline{auto\_service\_id}		& BIGINT	&  	Identifikator autoservisa koji pruža uslugu.	\\ \hline 
					
					
				\end{longtabu}
			
			\noindent\textbf{vehicle\_part} Ovaj entitet sadrži sve važne informacije za rezervni dio automobila. Ovaj entitet sadrži atribute: ID dijela, kilometraža za koju je dio predviđen, naziv dijela, njegova cijena te ID autoservisa koji ima rezervni dio. Ovaj entitet je u vezi \textit{Many-to-One} sa entitetom auto\_service preko ID-a tog autoservisa.
			
				\begin{longtabu} to \textwidth {|X[6, l]|X[6, l]|X[20, l]|}
					
					\hline \multicolumn{3}{|c|}{\textbf{vehicle\_part (Rezervni dio)}}	 \\[3pt] \hline
					\endfirsthead
					
					\hline \multicolumn{3}{|c|}{\textbf{vehicle\_part}}	 \\[3pt] \hline
					\endhead
					
					\hline 
					\endlastfoot
					
					\cellcolor{LightGreen}
					\textbf{part\_id} 				& BIGINT	&  	Identifikator rezervnog dijela. 	\\ \hline
					estimated\_ duration\_in\_km				& INTEGER 	&   Kilometraža za koju je predviđen rezervni dio.	\\ \hline 
					part\_name 				& VARCHAR 	& Naziv rezervnog dijela.  \\ \hline 
					price		& DOUBLE PRECISION	&  	Cijena rezervnog dijela.	\\ \hline 
					\underline{auto\_service\_id}		& BIGINT	&  	Identifikator autoservisa koji ima rezervni dio.	\\ \hline
					
					
				\end{longtabu}
			
			\subsection{Dijagram baze podataka}
%				\textit{ U ovom potpoglavlju potrebno je umetnuti dijagram baze podataka. Primarni i strani ključevi moraju biti označeni, a tablice povezane. Bazu podataka je potrebno normalizirati. Podsjetite se kolegija "Baze podataka".}
				
				\begin{figure}
					\centering
					\includegraphics[width=1.0\linewidth]{dijagrami/er-diagram}
					\caption{Relacijski model baze podataka}
					\label{fig:er-diagram}
				\end{figure}
			
		\eject
			
			
		\section{Dijagram razreda}
		
%			\textit{Potrebno je priložiti dijagram razreda s pripadajućim opisom. Zbog preglednosti je moguće dijagram razlomiti na više njih, ali moraju biti grupirani prema sličnim razinama apstrakcije i srodnim funkcionalnostima.}\\
%			
%			\textbf{\textit{dio 1. revizije}}\\
%			
%			\textit{Prilikom prve predaje projekta, potrebno je priložiti potpuno razrađen dijagram razreda vezan uz \textbf{generičku funkcionalnost} sustava. Ostale funkcionalnosti trebaju biti idejno razrađene u dijagramu sa sljedećim komponentama: nazivi razreda, nazivi metoda i vrste pristupa metodama (npr. javni, zaštićeni), nazivi atributa razreda, veze i odnosi između razreda.}\\
%			
%			\textbf{\textit{dio 2. revizije}}\\			
%			
%			\textit{Prilikom druge predaje projekta dijagram razreda i opisi moraju odgovarati stvarnom stanju implementacije}

		Na slici \ref{fig:classdiagrammodel} je prikazan dijagram razreda koji pripadaju sloju modela. Razred \textit{AppUser} predstavlja registriranog korisnika koji nakon unosa svojih podataka može koristiti osnovne funkcionalnosti aplikacije. Njega nasljeđuje razred \textit{Administrator} koji predstavlja administratora koji ima najveće ovlasti, razred \textit{VehicleOwner} koji predstavlja registriranog vlasnika automobila, razred \textit{ServiceEmployee} predstavlja osobu zaposlenu u auto servisu, koja može biti vlasnik tog auto servisa ili serviser ovisno ovisno o tipu (enumeracija \textit{ServiceEmployeeType}) te razred \textit{CurrentUser} u koji se pohranjuju podatci o trenutnom korisniku. 
		
		Razred \textit{Vehicle} predstavlja vozilo koje pripada nekom korisniku. Razred \textit{AutoService} označava jedan auto servis. Razred \textit{RepairOrder} predstavlja radni nalog za servis, ima trenutni status koji prikazuje je li nalog u čekanju, odbijen, prihvaćen ili gotov (enumeracija \textit{ServiceJobStatus}), te ima dva podrazreda, razred \textit{RegularRepairOrder} koji predstavlja radni nalog za redovni servis, te razred \textit{RepairingRepairOrder} koji predstavlja radni nalog za izvanredni servis. Razred \textit{RepairOrderItem} predstavlja konkretnu stavku (ugrađeni rezervni dio ili obavljenu uslugu) s cijenom na izvanrednom servisu. Razredi \textit{VehiclePart} i \textit{ServiceLabor} implementiraju sučelje \textit{Product} koji predstavlja proizvod ili uslugu. Razred \textit{VehiclePart} predstavlja rezervni dio za automobil, a razred \textit{ServiceLabor} predstavlja uslugu koju serviser može obaviti na automobilu te oni pripadaju određenom autoservisu te predstavljaju stavke koje se mogu dodati na radne naloge izvanrednog servisa.
		
		\begin{figure}[H]
			\centering
			\includegraphics[width=1.0\linewidth]{dijagrami/class_diagram_model.png}
			\caption{Dijagram razreda modela}
			\label{fig:classdiagrammodel}
		\end{figure}
	
		Na slici \ref{fig:classdiagramcontrolservice} dijagramom su prikazani razredi \textit{controller} i \textit{service}.
		
		Razred \textit{MainController} nudi operacije povezane s prikazom početne stranice te registraciju. Njegova sučelja su \textit{VehicleOwnerService}, \textit{AutoServiceService} i \textit{UserService}.
		
		Razred \textit{AdministratorController} nudi operacije povezane s administratorom. Njegova sučelja su \textit{VehicleOwnerService} i \textit{AutoServiceService}. 
		
		Razred \textit{VehicleOwnerController} nudi operacije povezane s vlasnikom vozila. Njegova sučelja su \textit{AutoServiceService}, \textit{VehicleService} i \textit{RepairOrderService}.
		
		Razred \textit{AutoServiceController} nudi operacije povezane sa serviserom. Njegova sučelja su \textit{RepairOrderService}, \textit{VehiclePartService}, \textit{ServiceLaborService} i \textit{ServiceEmployeeService}.
		
		Razred \textit{ApiController} je prazan.
		
		Razred \textit{GeoDTO} slući za postavljanje i dobavljanje lokacije te postavljanje i dobavljanje imena.
		
		Razred \textit{Utility} služi za provjeru OIB-a, e-maila, korisničkog imena, lozinke te ostalih izmijenjivih podataka i za provjeru postojanja podataka.
		
		Sučelje \textit{UserService} omogućava uređivanje računa svih korisnika. Sučelje \textit{VehicleOwnerService} omogućava uređivanje računa svih korisnika vlasnika vozila. Sučelje \textit{AutoServiceService} omogućava uređivanje svih autoservisa. Sučelje \textit{VehicleService} omogućava upravljanje vozilima. Sučelje \textit{RepairOrderService} omogućava upravljanje servisnim nalozima. Sučelje \textit{VehiclePartService} omogućava upravljanje zamjenskim komadima za vozila. Sučelje \textit{ServiceLaborService} omogućava upravljanje uslugama koje se nude. Sučelje \textit{ServiceEmployeeService} omogućava upravljanje radnicima (serviserima).
				
		
		\begin{figure}[H]
			\centering
			\includegraphics[width=1.0\linewidth]{dijagrami/class_diagram_controller_service.png}
			\caption{Dijagram razreda controller i service}
			\label{fig:classdiagramcontrolservice}
		\end{figure}
	
		Na slici \ref{fig:classdiagramimpl} prikazan je razred \textit{impl}.
	
		\begin{figure}[H]
			\centering
			\includegraphics[width=1.0\linewidth]{dijagrami/class_diagram_impl.png}
			\caption{Dijagram razreda impl}
			\label{fig:classdiagramimpl}
		\end{figure}
	
		Na slici \ref{fig:classdiagramhuorepositorysecurity} dijagramom su prikazani svi preostali razredi, preciznije razredi \textit{huo}, \textit{repository}, i \textit{security}. Razred \textit{MyAutoServiceApplication} predstavlja glavni razred od kojeg počinje izvršavanje poslužiteljske aplikacije. Razred \textit{Initialization} izvršava upis testnih podataka u bazu prilikom pokretanja aplikacije u razvojnom okruženju. Sučelja \textit{UserRepository} i \textit{AutoServiceRepository} pripadaju sloju pristupa podacima. Razredi \textit{UserDetailsServiceImpl} (čije je sučelje \textit{UserDetailsService}) i \textit{WebSecurity} služe za dohvaćanje podataka o korisniku prilikom autentikacije odnosno za konfiguraciju autentikacije i autorizacije, i podatke o trenutnom korisniku pohranjuju kao razred \textit{CurrentUser}.
	
		\begin{figure}[H]
			\centering
			\includegraphics[width=1.0\linewidth]{dijagrami/class_diagram_huo_repository_security.png}
			\caption{Dijagram razreda huo, repository i security}
			\label{fig:classdiagramhuorepositorysecurity}
		\end{figure}
		
			
			
			\eject
		
		\section{Dijagram stanja}
			
%			Slika \ref{fig:statediagram} prikazuje dijagram stanja koji opisuje kako se serviser služi aplikacijom. Nakon što mu se njegova prikaže početna stranica, on klikom na "Servisni nalozi" dobiva prikaz liste otvorenih naloga. Klikom na "Prihvati nalog" on prihvaća određeni nalog te, ovisno o tome je li nalog redovni ili izvanredni, biva preusmjeren na prozor predviđen taj nalog.
%			
%			Ako je nalog redovni, klikom na "Unos kilometraže" on unosi kilometražu vozila u trenutku servisa. Klikom na "Spremi" kilometraža se sprema te se serviser vraća na prikaz liste radnih naloga.
%			
%			Ako je nalog izvanredni, klikom na "Unos rezervnih dijelova" serviser se preusmjerava na iskočni prozor gdje, dok ih ima, unosi rezervne dijelove. Nakon što je završio, vraća se na prikaz naloga. Odabirom "Unos usluga" on na isti način unosi usluge koje su obavljene u servisu, dok ih ima. Klikom na "Zatvori" serviser se vraća na prikaz liste otvorenih naloga.
%			
%			Ukoliko na svojoj početnoj stranici serviser odabere opciju "Odjava", on se odjavljuje iz svog korisničkog računa.
			
%			\begin{figure}[H]
%				\centering
%				\includegraphics[width=1.0\linewidth]{dijagrami/state_diag_serviser}
%				\caption{Dijagram stanja za servisera}
%				\label{fig:statediagram}
%			\end{figure}
%			
%			\eject 
			
			Slika \ref{fig:statediagram} prikazuje dijagram stanja korisnika koji pristupa aplikaciji. On se na početnoj stranici može prijaviti kao serviser ili kao vlasnik automobila, ili može izaći iz preglednika. 
			
			Ukoliko se prijavi kao vlasnik automobila, prikazuje mu se popis automobila. Klikom na gumb "Detalji" prikazuju mu se radni nalozi za određeni automobil. Klikom na "Otvori nalog" on može otvoriti novi radni nalog, te ga nakon toga aplikacija vraća na popis radnih naloga za taj automobil. Također, vlasnik iz svih stanja može doći na prikaz dosadašnjih naloga, ili se odjaviti (tj. doći na početnu stranicu).
			
			Ako se korisnik prijavi kao serviser, prikazuje mu se lista pristiglih naloga koje on može prihvatiti klikom na gumb Prihvati (tada se nalog premješta u otvorene naloge) ili odbiti klikom na gumb Odbij (tada se nalog briše s liste). Klikom na "Otvoreni nalozi", servisera se preusmjerava na prikaz liste otvorenih naloga. Klikom na "Detalji" serviser može uređivati određeni nalog. Ukoliko je nalog za redovan servis, serviser klikom na "Završi" može završiti radni nalog, a klikom na "Povratak" vratiti se na listu. Ukoliko je nalog pak izvanredan, serviser može dodati novu uslugu klikom na "Dodaj uslugu" (tada mu se prikaže lista usluga) ili dodati novi rezervni dio klikom na "Dodaj rezervni dio" (tada mu se prikaže lista rezervnih dijelova). U oba slučaja, klik na "Dodaj" vraća servisera na stranicu tog naloga, te kao i kod redovitog radnog naloga, klikom na "Završi" može završiti radni nalog ili se vratiti na listu klikom na "Povratak". Također, serviser iz svih stanja može doći do prikaza liste pristiglih naloga, te se iz svih stanja može odjaviti.
			
			\begin{figure}[H]
				\centering
				\includegraphics[width=1.0\linewidth]{dijagrami/state-diagram-edited}
				\caption{Dijagram stanja za korisnika}
				\label{fig:statediagram}
			\end{figure}
		
			\eject 
			
%			Slika 4.10 prikazuje dijagram stanja vlasnika auta. Klikom na "Servisni nalozi" praikazuje se lista već postojećih naloga, nakon čega ima opciju otvoriti novi nalog.
%			
%			\begin{figure}[H]
%				\centering
%				\includegraphics[width=1.0\linewidth]{dijagrami/state_diag_carowner}
%				\caption{Dijagram stanja za vlasnika auta}
%				\label{fig:statediagram}
%			\end{figure}
%			
%			\eject
		
		\section{Dijagram aktivnosti}
			
			Slika \ref{fig:actdiag} prikazuje dijagram aktivnosti za životni tijek jednog radnog naloga, od otvaranja koje vrši korisnik do zatvaranja.
			
			Korisnik zatraži upravljanje nalozima web-aplikaciji, koja zatim šalje upit za dohvat podataka bazi podataka. Nakon što baza vrati podatke o automobilima i servisima, web-aplikacija prikazuje sučelje za otvaranje radnog naloga. Korisnik tada otvara novi radni nalog, a web-aplikacija šalje bazi podataka upit za dodavanjem podataka. Kada baza stvori novi radni nalog, web-aplikacija potvrđuje spremanje. 
			
			Nakon toga, serviser zatraži listu naloga te ju web-aplikacija dohvati iz baze. Serviser prihvaća radni nalog te ponovno preko web-aplikacije i baze ažurira status naloga. Nakon što web-aplikacija prikaže sučelje otvorenog naloga, serviser unosi kilometražu (postupak se ponavlja dok serviser ne unese ispravan format kilometraže, odnosno nenegativan broj), te odlučuje je li potreban izvanredni servis. Web-aplikacija šalje upit bazi koja ažurira podatke, nakon čega web-aplikacija prikazuje ažurirane podatke serviseru. 
			
			Naposljetku, serviser zatvara radni nalog, te web-aplikacija šalje upit za promjenom podataka bazi. Ona ažurira podatke, nakon čega web-aplikacija prikazuje potvrdu serviseru.
			 
			 \begin{figure}[H]
			 	\centering
			 	\includegraphics[width=1.0\linewidth]{dijagrami/activity-diagram-v2}
			 	\caption{Dijagram aktivnosti}
			 	\label{fig:actdiag}
			 \end{figure}
			
			\eject
		\section{Dijagram komponenti}
%		
%			\textbf{\textit{dio 2. revizije}}\\
%		
%			 \textit{Potrebno je priložiti dijagram komponenti s pripadajućim opisom. Dijagram komponenti treba prikazivati strukturu cijele aplikacije.}
		
		Na slici \ref{fig:componentdiagram} su je na dijagramu komponenti prikazana interna struktura i odnosi glavnih komponenti aplikacije te njihova međuovisnost s okolinom.
		
		Preglednik putem \textit{HTTP} zahtjeva i odgovora komunicira s aplikacijom. Ti zahtjevi u aplikaciji pristižu u interni Springov \textit{DispatcherServlet} koji pomoću komponente \textit{UserDetailsService} saznaje trenutnog korisnika i njegove ovlasti te prosljeđuje obradu komponenti \textit{Controllers} koja obuhvaća sve komponente u kontrolerskom sloju aplikacije. Pomoću vraćenih podataka i komponente \textit{Thymeleaf} iz predložaka se stvara \textit{HTML} kod koji \textit{DispatcherServler} šalje natrag klijentu.
		
		Komponenta \textit{Controllers} sve operacije obavlja oslanjajući se na komponentu \textit{JpaServices} koja implementira tražene funkcionalnosti. Ta komponenta dohvat podataka vrši iz repozitorija čije implementacije stvara komponenta \textit{Spring JPA} koja sama piše i parsira \textit{SQL} upite i odgovore te se spaja na bazu podataka.
		
		Postoji i mogućnost dohvata podataka o vozilu iz vanjskog HUO servisa koju interno vrši \textit{HUOConnector}, a koriste je ostali servisi. Funkcionalnost spajanja na vanjski HUO servis trenutno nije implementirana, pa \textit{HUO Connector} samo glumi tu funkcionalnost.
		
		

		\begin{figure}
			\centering
			\includegraphics[width=1.0\linewidth]{dijagrami/component_diagram}
			\caption{Dijagram komponenti}
			\label{fig:componentdiagram}
		\end{figure}

