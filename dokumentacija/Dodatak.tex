\chapter*{Dodatak: Prikaz aktivnosti grupe}
		\addcontentsline{toc}{chapter}{Dodatak: Prikaz aktivnosti grupe}
		
		\section*{Dnevnik sastajanja}
		
%		
%		\textbf{\textit{Kontinuirano osvježavanje}}\\
%		
%		 \textit{U ovom dijelu potrebno je redovito osvježavati dnevnik sastajanja prema predlošku.}
		
		\begin{packed_enum}
			\item sastanak
			\item[] \begin{packed_item}
				\item Datum: 09.10.2019.
				\item Prisustvovali: P.Matošević, K.Boras, D.V.Cvitković, D.Facković, M.Has, N.Kolarec, J.Prpić
				\item Teme sastanka:
				\begin{packed_item}
					\item  sastanak s asistentom
					\item  analiza zadanog zadatka i mogućih alata i tehnologija
					\item  rasprava potencijalnih ostalih tema
				\end{packed_item}
			\end{packed_item}
		
		\item sastanak
		\item[] \begin{packed_item}
			\item Datum: 17.10.2019.
			\item Prisustvovali: P.Matošević, K.Boras, D.V.Cvitković, D.Facković, N.Kolarec, J.Prpić
			\item Teme sastanka:
			\begin{packed_item}
				\item  postavljanje okoline potrebne za nastavak rada na projektu
				\item  upoznavanje sa Gitom i kontrolerima u Spring Boot-u
				\item  rasprava oko organizacije posla
				\item  diskusija o iskustvu članova tima u pojedinim tehnologijama
				\item  detaljna analiza zadatka
			\end{packed_item}
		\end{packed_item}
			
%			\item  sastanak
%			
%			\item[] \begin{packed_item}
%				\item Datum: u ovom formatu: \today
%				\item Prisustvovali: I.Prezime, I.Prezime
%				\item Teme sastanka:
%				\begin{packed_item}
%					\item  opis prve teme
%					\item  opis druge teme
%				\end{packed_item}
%			\end{packed_item}
%			
%			\item  sastanak
%			\item[] \begin{packed_item}
%				\item Datum: u ovom formatu: \today
%				\item Prisustvovali: I.Prezime, I.Prezime
%				\item Teme sastanka:
%				\begin{packed_item}
%					\item  opis prve teme
%					\item  opis druge teme
%				\end{packed_item}
%			\end{packed_item}
%			
			
		\end{packed_enum}
		
		\eject
		\section*{Tablica aktivnosti}
		
%			\textbf{\textit{Kontinuirano osvježavanje}}\\
%			
%			 \textit{Napomena: Doprinose u aktivnostima treba navesti u satima po članovima grupe po aktivnosti.}
%					
						
			
			\begin{longtabu} to \textwidth {|X[7, l]|X[1, c]|X[1, c]|X[1, c]|X[1, c]|X[1, c]|X[1, c]|X[1, c]|}
								
				\cline{2-8} \multicolumn{1}{c|}{\textbf{}} &     \multicolumn{1}{c|}{\rotatebox{90}{\textbf{Patrik Matošević }}} & \multicolumn{1}{c|}{\rotatebox{90}{\textbf{Katarina Boras }}} &	\multicolumn{1}{c|}{\rotatebox{90}{\textbf{Daria Vanesa Cvitković }}} &	\multicolumn{1}{c|}{\rotatebox{90}{\textbf{Dora Facković }}} &
				\multicolumn{1}{c|}{\rotatebox{90}{\textbf{Mislav Has }}} &
				\multicolumn{1}{c|}{\rotatebox{90}{\textbf{Nina Kolarec }}} &	\multicolumn{1}{c|}{\rotatebox{90}{\textbf{Juraj Prpić }}} \\ \hline 
				\endfirsthead
				
			
				\cline{2-8} \multicolumn{1}{c|}{\textbf{}} &     \multicolumn{1}{c|}{\rotatebox{90}{\textbf{Patrik Matošević }}} & \multicolumn{1}{c|}{\rotatebox{90}{\textbf{Katarina Boras }}} &	\multicolumn{1}{c|}{\rotatebox{90}{\textbf{Daria Vanesa Cvitković }}} &	\multicolumn{1}{c|}{\rotatebox{90}{\textbf{Dora Facković }}} &
				\multicolumn{1}{c|}{\rotatebox{90}{\textbf{Mislav Has }}} &
				\multicolumn{1}{c|}{\rotatebox{90}{\textbf{Nina Kolarec }}} &	\multicolumn{1}{c|}{\rotatebox{90}{\textbf{Juraj Prpić }}} \\ \hline
				\endhead
				
				
				\endfoot
							
				 
				\endlastfoot
				
				Upravljanje projektom 		&  &  &  &  &  &  & \\ \hline
				Opis projektnog zadatka 	&  &  &  &  &  &  & \\ \hline
				
				Funkcionalni zahtjevi       &  &  &  &  &  &  &  \\ \hline
				Opis pojedinih obrazaca 	&  &  &  &  &  &  &  \\ \hline
				Dijagram obrazaca 			&  &  &  &  &  &  &  \\ \hline
				Sekvencijski dijagrami 		&  &  &  &  &  &  &  \\ \hline
				Opis ostalih zahtjeva 		&  &  &  &  &  &  &  \\ \hline

				Arhitektura i dizajn sustava	 &  &  &  &  &  &  &  \\ \hline
				Baza podataka				&  &  &  &  &  &  &   \\ \hline
				Dijagram razreda 			&  &  &  &  &  &  &   \\ \hline
				Dijagram stanja				&  &  &  &  &  &  &  \\ \hline
				Dijagram aktivnosti 		&  &  &  &  &  &  &  \\ \hline
				Dijagram komponenti			&  &  &  &  &  &  &  \\ \hline
				Korištene tehnologije i alati 		&  &  &  &  &  &  &  \\ \hline
				Ispitivanje programskog rješenja 	&  &  &  &  &  &  &  \\ \hline
				Dijagram razmještaja			&  &  &  &  &  &  &  \\ \hline
				Upute za puštanje u pogon 		&  &  &  &  &  &  &  \\ \hline 
				Dnevnik sastajanja 			&  &  &  &  &  &  &  \\ \hline
				Zaključak i budući rad 		&  &  &  &  &  &  &  \\  \hline
				Popis literature 			&  &  &  &  &  &  &  \\  \hline
				&  &  &  &  &  &  &  \\ \hline \hline
				\textit{Dodatne stavke kako ste podijelili izradu aplikacije} 			&  &  &  &  &  &  &  \\ \hline
				\textit{npr. izrada početne stranice} 				&  &  &  &  &  &  &  \\ \hline 
				\textit{izrada baze podataka} 		 			&  &  &  &  &  &  & \\ \hline 
				\textit{spajanje s bazom podataka} 							&  &  &  &  &  &  &  \\ \hline
				\textit{back end} 							&  &  &  &  &  &  &  \\  \hline
				 							&  &  &  &  &  &  &\\  \hline
				
				
			\end{longtabu}
					
					
		\eject
		\section*{Dijagrami pregleda promjena}
		
		\textbf{\textit{dio 2. revizije}}\\
		
		\textit{Prenijeti dijagram pregleda promjena nad datotekama projekta. Potrebno je na kraju projekta generirane grafove s gitlaba prenijeti u ovo poglavlje dokumentacije. Dijagrami za vlastiti projekt se mogu preuzeti s gitlab.com stranice, u izborniku Repository, pritiskom na stavku Contributors.}
		
	