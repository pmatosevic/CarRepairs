\chapter{Zaključak i budući rad}
		
		% \textbf{\textit{dio 2. revizije}}\\
		
		% \textit{U ovom poglavlju potrebno je napisati osvrt na vrijeme izrade projektnog zadatka, koji su tehnički izazovi prepoznati, jesu li riješeni ili kako bi mogli biti riješeni, koja su znanja stečena pri izradi projekta, koja bi znanja bila posebno potrebna za brže i kvalitetnije ostvarenje projekta i koje bi bile perspektive za nastavak rada u projektnoj grupi.}
		
		 %\textit{Potrebno je točno popisati funkcionalnosti koje nisu implementirane u ostvarenoj aplikaciji.}
		 
		 Zadatak koji je naša grupa dobila sastojao se od izrade programske potpore za web-aplikaciju \textit{Moj AutoServis}. Ona bi vlasnicima automobila služila kao alat za praćenje stanja svojih vozila i izradu radnih naloga, a autoservisima omogućavala prihvat i obradu radnih naloga za vozila svojih klijenata, kao i upravljanje podatcima autoservisa. Na projektu smo radili nešto manje od četiri mjeseca, tijekom kojih smo se s obzirom na ciljeve projekta fokusirali više na dokumentaciju (prvi dio) ili na sam kod (drugi dio).
		 
		 Prvi dio izrade projekta sastojao se većinom od strukturiranja kako će sama aplikacija izgledati i prenošenja dogovorenih rješenja u dokumentaciju. Složili smo popis funkcionalnih zahtjeva, opisali sve potrebne obrasce uporabe te po njima izradili \textit{use-case} dijagrame, koji će kasnije biti ključni u izradi same aplikacije. Sekvencijski dijagrami pomogli su nam u shvaćanju procesa koji se moraju odviti u aplikaciji, a relacijski dijagram baze podataka te dijagram razreda pojasnio je \textit{backend}. Sklonost detaljima i vrijeme utrošeno na izradu ovih dijagrama uvelike nam je pomoglo kasnije, pogotovo u koordinaciji raznih dijelova aplikacije te izradi dizajna.
		 
		 Sa jasno napisanim ciljevima krenuli smo na samu izradu aplikacije. Iako su u timu postojale poveće razlike u programerskim vještinama, one su suradnjom i timskim radom smanjene, te je svaki član tima pridonio aplikaciji svojim sposobnostima. Članovi su stekli znanje o jezicima poput HTML-a, CSS-a i JavaScript-a, izradi testova u okviru Selenium, te naravno o strukturiranju aplikacije koja prati MVC-obrazac. U ovom dijelu izrade projekta dokumentirali smo i dijagrame stanja, aktivnosti, komponenti te razmještaja. Naposljetku smo dodali i upute za korištenje. 
		 
		 Sva komunikacija tima odvijala se preko Slack-a, koji je omogućio jasno odvajanje dretvi razgovora za određen dio aplikacije te lako slanje koda ili slika, kao i obavještvanje o napravljenim promjenama u aplikaciji na GitLab-u.
		 
		 Iako smo zadovoljni napravljenim, svaki rad uvijek može biti bolji. Neke od mogućnih nadogradnji uključuju:
		 \begin{packed_enum}
		 	\item Prilagodbu aplikacije za korištenje na mobilnim uređajima,
		 	\item Omogućavanje pregleda podataka o servisima automobila dok nije bio u vlasništvu trenutnog vlasnika,
		 	\item Omogućavanje ocjenjivanja kvalitete usluge nakon obavljenog servisa, i
		 	\item Elektroničko plaćanje usluge kroz aplikaciju.
		 \end{packed_enum}
	 
	 	Iako ponekad naporan i frustrirajuć -- bilo zbog vlastitog neznanja i neiskustva, bilo zbog nekoordinacije članova -- ovaj je projekt bio vrijedno iskustvo svim članovima \textit{InfiniTeam}-a. Kroz sve svoje izazove pružio nam je nešto novog iskustva, vještina, i malo bolju spremnost za sve nadolazeće projekte.
		
		\eject 