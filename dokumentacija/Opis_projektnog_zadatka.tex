\chapter{Opis projektnog zadatka}
		
		
		% \textbf{\textit{dio 1. revizije}}\\

		Cilj ovog projektnog zadatka je osmisliti i razviti programsku potporu za web-aplikaciju \textit{"MojAutoServis"} koja će registriranim vlasnicima automobila uštediti vrijeme prilikom obavljanja redovitih i izvanrednih servisa vozila. Aplikacija će ih brzo i efikasno povezati sa auto-servisima, tj. sa serviserima, te tako eliminirati potrebu za dugim i nepotrebnim čekanjem u redovima i za ispunjavanjem papirologije.
		
		
		
		% opis projektnog zadatka
		
		Kada neregistrirani korisnik pokrene aplikaciju, ima opciju pregledavanja svih auto-servisa te su mu dostupne informacije o njima. Također ima mogućnost registracije.
		Postoje tri vrste korisnika:
		\begin{packed_item}
			\item vlasnik automobila,
			\item auto-servis, i
			\item serviser.
		\end{packed_item}
		
		
		 Postojeći korisnik registrira se svojim jedinstvenim korisničkim imenom i lozinkom. Kako bi se registrirao, neregistrirani korisnik mora aplikaciji dati:
		\begin{packed_item}
			\item svoje ime,
			\item svoje prezime,
			\item OIB, te
			\item adresu svoje e-pošte.
		\end{packed_item}
		
		U slučaju da se radi o tvrtci, podatci koji su potrebni su:
		\begin{packed_item}
			\item naziv,
			\item adresa, i
			\item OIB tvrtke.
		\end{packed_item}
		
		Nakon registracije, korisnik unosi podatke o automobilu iz prometne dozvole, među kojima su obavezno:
		\begin{packed_item}
			\item registracijska oznaka, te
			\item marka automobila
		\end{packed_item}
		
		Korisnik može u bilo kojem trenutku dodati još automobila, te ih može imati neograničeno mnogo. Kako bi novi automobil bio dodan, uneseni podaci moraju biti potvrđeni u HUO registru. Ukoliko to želi, korisnik također ima mogućnost izbrisati automobil.
		 
		Kada korisnik želi poslati svoj automobil na servis, mora otvoriti radni nalog, prilikom čega odabire redoviti ili izvanredni servis. Nakon što je rad na automobilu obavljen, korisniku dolaze detalji servisa, te su mu vidljivi zajedno sa detaljima svih prijašnjih obavljenih servisa. Te informacije su dostupne sve dok se automobil ne obriše iz aplikacije.
		
		Auto-servisi nakon registracije dodaju korisnički račun svoga administratora te servisere. Potom dodaju cjenik usluga i podatke o auto-dijelovima, te započinju s radom. Dostupna im je lista servisera preko koje mogu upravljati svojim zaposlenicima.  
		
		Serviseri primaju radne naloge poslane njihovoj tvrtci. Ako ih prihvate obavljaju servis, te u radni nalog dopisuju dodatne podatke******. Naknadno, ako je na automobilu primjećena potreba za dodatnim popravcima, serviser daje preporuku za izvanredan servis. Serviseri, kao i obični korisnici, mogu dodavati podatke o vlastitim automobilima i sami otvarati radne naloga za njih.
		
		Glavni administrator odobrava registracije autoservisa, ima upravu nad svim korisnicima, može brisati račune i ima pristup svim statističkim podatcima koji se izračunavaju. \textbf{\texttt{\textcolor{red}{Treba dovršiti/dopuniti}}} \\
		
		\newpage
		\Large Primjeri sličnih rješenja
		
		\normalsize Dvije aplikacije koje su najsličnije ovoj su ShopBoss i GetAFix. 
		
		% slične aplikacije
		GetAFix je aplikacija za upravljanje auto-servisom. Za razliku od ove aplikacije, korisnika se traži da na slici automobila označi gdje postoje ogrebotine, te da procijeni stanje auta u raznim kategorijama (ulje, \ldots).
		
		\begin{figure}[H]
			\includegraphics[scale=0.2]{slike/getafix.JPG}
			\centering
			\caption{GetAFix}
			\label{fig:idk}
		\end{figure}  
		
		ShopBoss je software dizajniran za vlasnike automobila i auto-servise. Ima više mogućnosti od aplikacije MojAutoServis, te uz to ima mogućnost pristupa korisničkom računu sa bilo kojeg uređaja. U nju su također integrirani razni alati i web-stranice poput Carfax-a itd.
		
		\begin{figure}[H]
			\includegraphics[scale=0.3]{slike/shopboss.PNG}
			\centering
			\caption{ShopBoss}
			\label{fig:idk1}
		\end{figure}
		
		Uz ove aplikacije, postoje mnoge koje su usmjerene samo na vlasnike automobila, pružajući im jednostavno i efikasno praćenje stanja njihovih vozila, te one koje su usmjerene samo na auto-servise i pomažu im u organizaciji i provođenju servisa. Ova aplikacija ta će dva smjera spojiti u jedan, omogućavajući i vlasnicima automobila i auto-servisima da komuniciraju i organiziraju servise automobila. \\
		
			\Large Moguće nadogradnje projektnog zadatka
		
		\normalsize Jedna od mogućih nadogradnji koje bi učinile ovu web-stranicu pristupačnijom je prilagodba iste za korištenje na mobilnim uređajima. Ovo bi bilo izrazito pogodno za slučajeve kada je korisniku potreban izvandredni servis zbog kvara koji se dogodio tijekom vožnje, npr. nakon sudara. Web-stranica bi mogla biti dalje nadograđena da nudi uslugu vučne službe,
		
		Još jedan potencijalni dodatak projektnom zadatku je mogućnost korisnika koji dodaje novi automobil da vidi podatke o svim obavljenim servisima nad tim automobilom dok nije bio u njegovom vlasništvu.
		
		
		
		
		autopraonica?
		ponuda rezervnih dijelova i njihove ugradnje
		rating
		
		
		
		
		
		hhhh\\
		\eject
		
		\section{Primjeri u LaTeXu}
		
		\textit{Ovo potpoglavlje izbrisati.}\\

		U nastavku se nalaze različiti primjeri kako koristiti osnovne funkcionalnosti LaTeXa koje su potrebne za izradu dokumentacije. Za dodatnu pomoć obratiti se asistentu na projektu ili potražiti upute na sljedećim web sjedištima:
		\begin{itemize}
			\item Upute za izradu diplomskog rada u LaTeXu - \url{https://www.fer.unizg.hr/_download/repository/LaTeX-upute.pdf}
			\item LaTeX projekt - \url{https://www.latex-project.org/help/}
			\item StackExchange za Tex - \url{https://tex.stackexchange.com/}\\
		
		\end{itemize} 	


		
		%Ovo poglavlje je potrebno prilikom predaje obrisati
		
		\underbar{podcrtani tekst}, 
		\textbf{podebljani tekst}, 
		\textit{nagnuti tekst}\\
		\normalsize primjer
		\large primjer
		\Large primjer
		\LARGE {primjer}
		\huge {primjer}
		\Huge primjer
		\normalsize
				
		\begin{packed_item}
			
			\item  primjer
			\item  primjer
			\item  primjer
			\item[] \begin{packed_enum}
				
				\item primjer
				\item primjer
			\end{packed_enum}
			
		\end{packed_item}
		
		\noindent primjer url-a: \url{https://www.fer.unizg.hr/predmet/opp/projekt}
		
		
		\begin{longtabu} to \textwidth {|X[8, l]|X[8, l]|X[16, l]|} %definicija sirine polja
			
			\hline \multicolumn{3}{|c|}{\textbf{naslov unutar tablice}}	 \\[3pt] \hline
			\endfirsthead
			
			\hline \multicolumn{3}{|c|}{\textbf{naslov unutar tablice}}	 \\[3pt] \hline
			\endhead
			
			\hline 
			\endlastfoot
			
			\rowcolor{LightGreen}IDKorisnik & INT	&  	Lorem ipsum dolor sit amet, consectetur adipiscing elit, sed do eiusmod  	\\ \hline
			korisnickoIme	& VARCHAR &   	\\ \hline 
			email & VARCHAR &   \\ \hline 
			ime & VARCHAR	&  		\\ \hline 
			\cellcolor{LightBlue} primjer	& VARCHAR &   	\\ \hline 
			
			
		\end{longtabu}
		

		\begin{table}[H]
			
			
			
			\begin{longtabu} to \textwidth {|X[8, l]|X[8, l]|X[16, l]|} %definicija sirine polja
				
				\hline 
				\endfirsthead
				
				\hline 
				\endhead
				
				\hline 
				\endlastfoot
				
				\rowcolor{LightGreen}IDKorisnik & INT	&  	Lorem ipsum dolor sit amet, consectetur adipiscing elit, sed do eiusmod  	\\ \hline
				korisnickoIme	& VARCHAR &   	\\ \hline 
				email & VARCHAR &   \\ \hline 
				ime & VARCHAR	&  		\\ \hline 
				\cellcolor{LightBlue} primjer	& VARCHAR &   	\\ \hline 
				
				
			\end{longtabu}
	
			\caption{\label{tab:referencatablica} Naslov ispod tablice.}
		\end{table}
		
		\begin{figure}[H]
			\includegraphics[scale=0.4]{slike/aktivnost.PNG}
			\centering
			\caption{Primjer slike s potpisom}
			\label{fig:promjene}
		\end{figure}
		
		\begin{figure}[H]
			\includegraphics[width=\linewidth]{slike/aktivnost.PNG}
			\caption{Primjer slike s potpisom 2}
			\label{fig:promjene2}
		\end{figure}
		
		
		
		\eject
		
	