\chapter{Opis projektnog zadatka}
		
		
		% \textbf{\textit{dio 1. revizije}}\\

		
		Cilj ovog projektnog zadatka je osmisliti i razviti programsku potporu  za web aplikaciju \textit{"Moj AutoServis"},  koja će registriranim korisnicima aplikacije --- vlasnicima automobila s jedne i auto servisima sa druge strane --- omogućiti praćenje redovitog i izvanrednog servisiranja automobila. Aplikacija će brzo i efikasno povezati  vlasnike automobila sa auto servisima te eliminirati potrebu za dugim i nepotrebnim čekanjem, odnosno skratiti vrijeme od narudžbe do realizacije. 
		
		
		
		% opis projektnog zadatka
		
		Aplikaciju će moći koristiti samo registrirani korisnici. Registrirani korisnik prijavljuje se sa svojim jedinstvenim korisničkim imenom i lozinkom. Postoje tri vrste korisnika:
		\begin{packed_item}
			\item vlasnik automobila,
			\item auto-servis, i
			\item serviser.
		\end{packed_item}
		
		
		 Kako bi se registrirao, vlasnik automobila mora aplikaciji dati sljedeće podatke:
		\begin{packed_item}
			\item svoje ime,
			\item svoje prezime,
			\item OIB, te
			\item adresu svoje e-pošte.
		\end{packed_item}
		
		U slučaju da se radi o tvrtci, podatci koji su potrebni su:
		\begin{packed_item}
			\item naziv,
			\item adresa, i
			\item OIB tvrtke.
		\end{packed_item}
		
		Nakon registracije, korisnik unosi podatke o automobilu iz prometne dozvole, među kojima su obavezno:
		\begin{packed_item}
			\item registracijska oznaka, te
			\item marka automobila.
		\end{packed_item}
		
		Kako bi novi automobil bio dodan, uneseni podaci  se provjeravaju u HUO registru. U slučaju nepodudaranja podataka  vozila sa podacima u HUO registru, odbija se registracija vozila u aplikaciji. Korisnik može u bilo kojem trenutku dodati još automobila, te  može unijeti podatke o  neograničeno mnogo vozila. Ukoliko to želi, korisnik također ima mogućnost  izbrisati podatke o automobilu. 
		
		Auto servisi također moraju stvoriti svoje profile u aplikaciji, odnosno moraju se registrirati kao korisnici sa podacima:
		\begin{packed_item}
			\item naziv,
			\item adresa, i
			\item OIB.
		\end{packed_item}
		 
		 Nakon registracije, auto servisi trebaju definirati  korisnički račun za administratora auto servisa i za servisere. Administrator auto servisa potom unosi cjenik usluga sa podacima:
		 \begin{packed_item}
		 	\item vrsta usluge, i
		 	\item cijena usluge,
		 \end{packed_item}
	 
	 	 te podatke o auto-dijelovima, koji se ugrađuju u servisu:
	 	 \begin{packed_item}
	 	 	\item naziv dijela,
	 	 	\item rok trajanja,
	 	 	\item predviđena kilometraža, i
	 	 	\item cijena dijela,
	 	 \end{packed_item}
		 
		 kao i listu servisera sa podacima, preko koje mogu upravljati svojim zaposlenicima. 
		
		Kada korisnik (vlasnik automobila) želi poslati svoj automobil na servis, mora otvoriti radni nalog, prilikom čega odabire auto servis u kojem želi obaviti servis. Odabire redoviti ili izvanredni servis. Nakon što je rad na automobilu obavljen, korisniku dolaze detalji servisa sa cijenom.  
		
		Aplikacija nudi vlasniku automobila različite statističke podatka o troškovima, zamijenjenim dijelovima i drugim aspektima svih prijašnjih obavljenih servisa. Te informacije su dostupne sve dok se automobil ne obriše iz aplikacije.
		
		Serviseri primaju radne naloge poslane njihovoj tvrtci. 
		Ako ih prihvate, obavljaju servis, te u radni nalog dopisuju dodatne podatke o servisiranju, kao što su kilometraža, zamjena ulja, filtera i sl. Naknadno, ako je na automobilu primijećena potreba za dodatnim popravcima, serviser daje preporuku za izvanredni servis. U slučaju izvanrednog servisa unose se podaci o kvarovima, popravcima kao i ugrađenim rezervnim dijelovima.  
		
		Serviseri, kao i obični korisnici, mogu dodavati podatke o vlastitim automobilima i sami otvarati radne naloge za njih.

		Aplikacija nudi auto servisu različite statičke podatke o troškovima servisa, stanju rezervnih dijelova, te financijskim aspektima poslovanja. Također, nudi i statičke podatke o serviserima.
		
		Glavni administrator odobrava registracije auto servisa, ima upravu nad svim korisnicima, može brisati račune i ima pristup svim statističkim podatcima koji se izračunavaju. 
		
		
		\newpage
		\Large Primjeri sličnih rješenja
		
		\normalsize Dvije aplikacije koje su najsličnije ovoj su ShopBoss i GetAFix. 
		
		% slične aplikacije
		GetAFix je aplikacija za upravljanje auto servisom. Za razliku od projektne aplikacije, korisnika se traži da na slici automobila označi gdje postoje ogrebotine, te da procijeni stanje auta u raznim kategorijama (ulje, \ldots).
		
		\begin{figure}[H]
			\includegraphics[scale=0.2]{slike/getafix.JPG}
			\centering
			\caption{GetAFix}
			\label{fig:idk}
		\end{figure}  
		
		ShopBoss je software dizajniran za vlasnike automobila i auto-servise. Ima više mogućnosti od aplikacije \textit{Moj AutoServis}, te uz to ima mogućnost pristupa korisničkom računu sa bilo kojeg uređaja. U nju su također integrirani razni alati i web-stranice poput Carfax-a itd.
		
		\begin{figure}[H]
			\includegraphics[scale=0.3]{slike/shopboss.PNG}
			\centering
			\caption{ShopBoss}
			\label{fig:idk1}
		\end{figure}
		
		Uz ove aplikacije, postoje mnoge koje su usmjerene samo na vlasnike automobila, pružajući im jednostavno i efikasno praćenje stanja njihovih vozila, te one koje su usmjerene samo na auto-servise i pomažu im u organizaciji i provođenju servisa. Ova aplikacija ta će dva smjera spojiti u jedan, omogućavajući i vlasnicima automobila i auto servisima da komuniciraju i organiziraju servise automobila. \\
		
			\Large Moguće nadogradnje projektnog zadatka
		
		\normalsize Jedna od mogućih nadogradnji koje bi učinile ovu aplikaciju pristupačnijom je prilagodba iste za korištenje na mobilnim uređajima. Ovo bi bilo izrazito pogodno za slučajeve kada je korisniku potreban izvandredni servis zbog kvara koji se dogodio tijekom vožnje, npr. nakon sudara. Aplikacija bi mogla biti dalje nadograđena da nudi uslugu vučne službe, tj. da omogućuje korisniku da na karti označi gdje se nalazi sa svojim vozilom, koje bi potom bilo odvučeno na servis.
		
		Još jedan potencijalni dodatak projektnom zadatku je omogućavanje korisniku da, nakon što doda novi automobil, ima pristup podacima o svim servisima nad tim automobilom obavljenim dok nije bio u njegovom vlasništvu.
		
		Naknadno, moguće je nadograditi aplikaciju na način da svaki korisnik ima priliku ocijeniti kvalitetu usluge nakon obavljenog servisa. Tada bi se mogla izračunati prosječna ocjena dodjeljena pojedinom serviseru, te pojedinom auto servisu u cijelosti. Te bi ocjene bile vidljive korisniku dok odabire gdje želi da je njegov automobil servisiran.
		
		Naposljetku, još jedna ideja za nadogradnju je mogućnost da vlasnik automobila može platiti za uslugu elektronički kroz aplikaciju, što bi uštedilo još vrememna kod podizanja automobila iz auto servisa.\\
		\eject
		
		\section{Primjeri u LaTeXu}
		
		\textit{Ovo potpoglavlje izbrisati.}\\

		U nastavku se nalaze različiti primjeri kako koristiti osnovne funkcionalnosti LaTeXa koje su potrebne za izradu dokumentacije. Za dodatnu pomoć obratiti se asistentu na projektu ili potražiti upute na sljedećim web sjedištima:
		\begin{itemize}
			\item Upute za izradu diplomskog rada u LaTeXu - \url{https://www.fer.unizg.hr/_download/repository/LaTeX-upute.pdf}
			\item LaTeX projekt - \url{https://www.latex-project.org/help/}
			\item StackExchange za Tex - \url{https://tex.stackexchange.com/}\\
		
		\end{itemize} 	


		
		%Ovo poglavlje je potrebno prilikom predaje obrisati
		
		\underbar{podcrtani tekst}, 
		\textbf{podebljani tekst}, 
		\textit{nagnuti tekst}\\
		\normalsize primjer
		\large primjer
		\Large primjer
		\LARGE {primjer}
		\huge {primjer}
		\Huge primjer
		\normalsize
				
		\begin{packed_item}
			
			\item  primjer
			\item  primjer
			\item  primjer
			\item[] \begin{packed_enum}
				
				\item primjer
				\item primjer
			\end{packed_enum}
			
		\end{packed_item}
		
		\noindent primjer url-a: \url{https://www.fer.unizg.hr/predmet/opp/projekt}
		
		
		\begin{longtabu} to \textwidth {|X[8, l]|X[8, l]|X[16, l]|} %definicija sirine polja
			
			\hline \multicolumn{3}{|c|}{\textbf{naslov unutar tablice}}	 \\[3pt] \hline
			\endfirsthead
			
			\hline \multicolumn{3}{|c|}{\textbf{naslov unutar tablice}}	 \\[3pt] \hline
			\endhead
			
			\hline 
			\endlastfoot
			
			\rowcolor{LightGreen}IDKorisnik & INT	&  	Lorem ipsum dolor sit amet, consectetur adipiscing elit, sed do eiusmod  	\\ \hline
			korisnickoIme	& VARCHAR &   	\\ \hline 
			email & VARCHAR &   \\ \hline 
			ime & VARCHAR	&  		\\ \hline 
			\cellcolor{LightBlue} primjer	& VARCHAR &   	\\ \hline 
			
			
		\end{longtabu}
		

		\begin{table}[H]
			
			
			
			\begin{longtabu} to \textwidth {|X[8, l]|X[8, l]|X[16, l]|} %definicija sirine polja
				
				\hline 
				\endfirsthead
				
				\hline 
				\endhead
				
				\hline 
				\endlastfoot
				
				\rowcolor{LightGreen}IDKorisnik & INT	&  	Lorem ipsum dolor sit amet, consectetur adipiscing elit, sed do eiusmod  	\\ \hline
				korisnickoIme	& VARCHAR &   	\\ \hline 
				email & VARCHAR &   \\ \hline 
				ime & VARCHAR	&  		\\ \hline 
				\cellcolor{LightBlue} primjer	& VARCHAR &   	\\ \hline 
				
				
			\end{longtabu}
	
			\caption{\label{tab:referencatablica} Naslov ispod tablice.}
		\end{table}
		
		\begin{figure}[H]
			\includegraphics[scale=0.4]{slike/aktivnost.PNG}
			\centering
			\caption{Primjer slike s potpisom}
			\label{fig:promjene}
		\end{figure}
		
		\begin{figure}[H]
			\includegraphics[width=\linewidth]{slike/aktivnost.PNG}
			\caption{Primjer slike s potpisom 2}
			\label{fig:promjene2}
		\end{figure}
		
		
		
		\eject
		
	