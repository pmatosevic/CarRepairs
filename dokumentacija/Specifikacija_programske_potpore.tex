\chapter{Specifikacija programske potpore}
		
	\section{Funkcionalni zahtjevi}

			\noindent \textbf{Dionici:}

			\begin{packed_enum}

                \item Naručitelj (FER)
				\item Razvojni tim
				\item Vlasnici automobila
				\item Vlasnici autoservisa
				\item Serviseri
				\item Administrator

			\end{packed_enum}

			\noindent \textbf{Aktori i njihovi funkcionalni zahtjevi:}


			\begin{packed_enum}

				\item  \underbar{Posjetitelj (inicijator) može:}

				\begin{packed_enum}

					\item Registracija vlasnika automobila
					\item Registracija vlasnika autoservisa
					\item Pregled autoservisa

				\end{packed_enum}

				\item \underbar{Registrirani korisnik (inicijator) može:}

				\begin{packed_enum}

					\item Upravljanje korisničkim računom
					\begin{packed_enum}

						\item Pregled korisničkih podataka
						\item Promjena korisničkih podataka

					\end{packed_enum}

					\item Prijava u sustav
					\item Odjava

				\end{packed_enum}

				\item  \underbar{Vlasnik automobila (inicijator) može:}

				\begin{packed_enum}

					\item Pregled autoservisa
					\item Upravljanje automobilima
					\begin{packed_enum}

						\item Dodavanje automobila
						\item Pregled statističkih podataka za automobil
						\item Brisanje automobila

					\end{packed_enum}

					\item Upravljanje servisnim nalozima
					\begin{packed_enum}

						\item Otvaranje novog servisnog naloga
						\item Pregled statusa servisnog naloga

					\end{packed_enum}

				\end{packed_enum}

				\item  \underbar{Vlasnik autoservisa (inicijator) može:}

				\begin{packed_enum}

					\item Upravljanje rezervnim dijelovima
					\begin{packed_enum}

						\item Dodavanje rezervnog dijela
						\item Uređivanje rezervnog dijela
						\item Brisanje rezervnog dijela

					\end{packed_enum}
				
					\item Upravljanje cjenikom usluga
					\begin{packed_enum}
						
						\item Dodavanje usluge
						\item Uređivanje usluge
						\item Brisanje usluge
						
					\end{packed_enum}

					\item Uređivanje podataka autoservisa
					\item Upravljanje serviserima
					\begin{packed_enum}

						\item Dodavanje servisera
						\item Uklanjanje servisera

					\end{packed_enum}

				\end{packed_enum}

				\item  \underbar{Serviser (inicijator) može:}

				\begin{packed_enum}
					
					\item Prihvaćanje radnog naloga
					\item Zatvaranje radnog naloga
					\item Rad na radnom nalogu
					\begin{packed_enum}

						\item Uređivanje naloga redovnog servisa
						\item Rad na nalogu izvanrednog servisa
						\begin{packed_enum}

							\item Dodavanje ugrađenog rezervnog dijela
							\item Brisanje ugrađenog rezervnog dijela
							\item Dodavanje usluge
							\item Brisanje usluge

						\end{packed_enum}

					\end{packed_enum}

				\end{packed_enum}

				\item  \underbar{Administrator (inicijator) može:}

				\begin{packed_enum}

					\item Administracija korisnika
					\begin{packed_enum}

						\item Pregled podataka korisničkog računa
						\item Brisanje korisničkog računa

					\end{packed_enum}

					\item Administracija autoservisa
					\begin{packed_enum}

						\item Pregled podataka autoservisa
						\item Brisanje autoservisa

					\end{packed_enum}

				\end{packed_enum}

				\item  \underbar{HUO registar (sudionik) može:}

				\begin{packed_enum}

					\item Provjera registracije i dohvat broja šasije

				\end{packed_enum}

				\item \underbar{Baza podataka (sudionik) može:}

				\begin{packed_enum}

					\item Komunikacija sa svim dijelovima sustava
					\item Pohrana podataka i ovlasti

				\end{packed_enum}

			\end{packed_enum}
			
			\eject 
			
			
				
			\subsection{Obrasci uporabe}
				
%				\textbf{\textit{dio 1. revizije}}
%				
%				\subsubsection{Opis obrazaca uporabe}
%					\textit{Funkcionalne zahtjeve razraditi u obliku obrazaca uporabe. Svaki obrazac je potrebno razraditi prema donjem predlošku. Ukoliko u nekom koraku može doći do odstupanja, potrebno je to odstupanje opisati i po mogućnosti ponuditi rješenje kojim bi se tijek obrasca vratio na osnovni tijek.}\\
					


\noindent \underbar{\textbf{UC1 -Registracija vlasnika automobila}}
\begin{packed_item}

	\item \textbf{Glavni sudionik: } Posjetitelj
	\item  \textbf{Cilj:} Stvaranje novog korisničkog računa vlasnika
	automobila
	\item  \textbf{Sudionici:} Baza podataka
	\item  \textbf{Preduvjet:} -
	\item  \textbf{Opis osnovnog tijeka:}

	\item[] \begin{packed_enum}

		\item Posjetitelj odabire opciju registracije
		\item Posjetitelj unosi osobne podatke i nastavlja registraciju
		\item Sustav provjerava ispravnost te ih sprema u bazu podataka ako su ispravni
		\item Posjetitelju se prikazuje poruka o uspješnoj registraciji
	\end{packed_enum}

	\item  \textbf{Opis mogućih odstupanja:}

	\item[] \begin{packed_item}

		\item[3.a] Posjetitelj je unio neispravan OIB, korisničko ime ili e-mail,
		ili postoji registrirani korisnik s nekim od tih podataka
		\item[] \begin{packed_enum}

			\item Posjetitelju se prikazuje poruka o neispravnosti/zauzetosti
			podataka

		\end{packed_enum}

	\end{packed_item}
\end{packed_item}

\noindent \underbar{\textbf{UC2 -Registracija vlasnika autoservisa}}
\begin{packed_item}
	\item \textbf{Glavni sudionik: } Posjetitelj
	\item  \textbf{Cilj:} Stvaranje novog korisničkog računa vlasnika
	autoservisa
	\item  \textbf{Sudionici:} Baza podataka
	\item  \textbf{Preduvjet:} -
	\item  \textbf{Opis osnovnog tijeka:}

	\item[] \begin{packed_enum}

		\item Posjetitelj odabire opciju registracije autoservisa
		\item Posjetitelj unosi podatke o tvrtki te svoje korisničke podatke za
		vlasnika servisa
		\item Sustav provjerava upisane podatke te ih sprema u bazu podataka ako su ispravni
		\item Posjetitelju se prikazuje poruka o uspješnoj registraciji
	\end{packed_enum}

	\item  \textbf{Opis mogućih odstupanja:}

	\item[] \begin{packed_item}

		\item[3.a] Posjetitelj je unio neispravan OIB ili korisničko ime ili već
		postoji tvrtka s nekim od tih podataka
		\item[] \begin{packed_enum}

			\item Posjetitelju se prikazuje poruka o neispravnosti/zauzetosti
			podataka

		\end{packed_enum}

	\end{packed_item}
\end{packed_item}

\noindent \underbar{\textbf{UC3 -Prijava u sustav}}
\begin{packed_item}

	\item \textbf{Glavni sudionik: } Registrirani korisnik
	\item  \textbf{Cilj:} Dobiti pristup korisničkom sučelju
	\item  \textbf{Sudionici:} Baza podataka
	\item  \textbf{Preduvjet:} Korisnik je registriran kao vlasnik automobila
	ili serviser
	\item  \textbf{Opis osnovnog tijeka:}

	\item[] \begin{packed_enum}

		\item Korisnik odabire opciju prijave u aplikaciju
		\item Korisnik upisuje korisničko ime i lozinku i nastavlja prijavu
		\item Sustav provjerava upisane podatke u bazi podataka
		\item Korisnika se preusmjerava na korisničko sučelje ovisno o tome je li
		vlasnik automobila ili serviser
	\end{packed_enum}

	\item  \textbf{Opis mogućih odstupanja:}

	\item[] \begin{packed_item}

		\item[3.a] Korisnik je unio neispravno korisničko ime ili lozinku
		\item[] \begin{packed_enum}

			\item Korisniku se prikazuje poruka o neispravnosti, a korisnik može
			pokušati ispraviti podatke

		\end{packed_enum}

	\end{packed_item}
\end{packed_item}

\noindent \underbar{\textbf{UC4 -Odjava}}
\begin{packed_item}

	\item \textbf{Glavni sudionik: } Registrirani korisnik
	\item  \textbf{Cilj:} Odjaviti se iz korisničkog sučelja
	\item  \textbf{Sudionici:} -
	\item  \textbf{Preduvjet:} Korisnik je prijavljen u sustav
	\item  \textbf{Opis osnovnog tijeka:}

	\item[] \begin{packed_enum}

		\item Korisnik odabire opciju odjave iz aplikacije
		\item Sustav odjavljuje korisnika te ga preusmjeri na početnu stranicu
	\end{packed_enum}

\end{packed_item}

\noindent \underbar{\textbf{UC5 -Pregled autoservisa}}
\begin{packed_item}

	\item \textbf{Glavni sudionik: } Posjetitelj, vlasnik automobila
	\item  \textbf{Cilj:} Pregled svih dostupnih autoservisa u aplikaciji
	\item  \textbf{Sudionici:} Baza podataka
	\item  \textbf{Preduvjet:} -
	\item  \textbf{Opis osnovnog tijeka:}

	\item[] \begin{packed_enum}

		\item Vlasnik automobila/posjetitelj odabire opciju za pregled svih autoservisa na početnom sučelju aplikacije
		\item Sustav dohvati iz baze listu svih registriranih autoservisa i prikaže ih
	\end{packed_enum}

\end{packed_item}

\noindent \underbar{\textbf{UC6 -Pregled korisničkih podataka}}
\begin{packed_item}

	\item \textbf{Glavni sudionik: } Registrirani korisnik
	\item  \textbf{Cilj:} Pregled korisničkih podataka
	\item  \textbf{Sudionici:} Baza podataka
	\item  \textbf{Preduvjet:} Korisnik je prijavljen u sustav
	\item  \textbf{Opis osnovnog tijeka:}

	\item[] \begin{packed_enum}

		\item Korisnik klikne odabire opciju upravljanja korisničkim računom
		\item Sustav prikaže osnovne podatke korisnika, nakon što ih dohvati iz baze
	\end{packed_enum}

\end{packed_item}

\noindent \underbar{\textbf{UC7 -Promjena korisničkih podataka}}
\begin{packed_item}

	\item \textbf{Glavni sudionik: } Registrirani korisnik
	\item  \textbf{Cilj:} Promjena korisničkih podataka
	\item  \textbf{Sudionici:} Baza podataka
	\item  \textbf{Preduvjet:} Korisnik je prijavljen u sustav
	\item  \textbf{Opis osnovnog tijeka:}

	\item[] \begin{packed_enum}

		\item Korisnik odabire opciju upravljanja korisničkim računom
		\item Sustav dozvoli promjenu i prikaže sučelje za promjene korisničkih
		podataka
		\item Korisnik promijeni podatke i potvrđuje promjene lozinkom
		\item Sustav sprema nove podatke u bazu i prikazuje poruku za uspješnu
		promjenu podataka
	\end{packed_enum}

\end{packed_item}

\noindent \underbar{\textbf{UC8 -Dodavanje automobila}}
\begin{packed_item}

	\item \textbf{Glavni sudionik: } Vlasnik automobila
	\item  \textbf{Cilj:} Dodavanje automobila
	\item  \textbf{Sudionici:} Baza podataka, HUO registar
	\item  \textbf{Preduvjet:} Korisnik je prijavljen u sustav kao vlasnik automobila
	\item  \textbf{Opis osnovnog tijeka:}

	\item[] \begin{packed_enum}

		\item Vlasnik automobila odabire opciju upravljanja svojim automobila
		\item Sustav prikaže sučelje za dodavanje novog automobila
		\item Vlasnik automobila unosi registracijsku oznaku i marku automobila
		\item Sustav provjerava unesene podatke pomoću HUO registra i dohvaća broj šasije, te sprema podatke u bazu
		\item Vlasniku automobila se prikazuje poruka o uspješnom dodavanju automobila

	\end{packed_enum}

	\item  \textbf{Opis mogućih odstupanja:}

	\item[] \begin{packed_item}

		\item[4.a] Vlasnik automobila je unio neispravnu registracijsku oznaku
		\item[] \begin{packed_enum}

			\item Vlasniku automobila se prikazuje poruka o neispravnosti i može
			pokušati ispraviti podatke

		\end{packed_enum}

	\end{packed_item}
\end{packed_item}

\noindent \underbar{\textbf{UC9 -Pregled statističkih podataka za
		automobil}}
\begin{packed_item}

	\item \textbf{Glavni sudionik: } Vlasnik automobila
	\item  \textbf{Cilj:} Pregled podataka o odabranom automobilu
	\item  \textbf{Sudionici:} Baza podataka
	\item  \textbf{Preduvjet:} Korisnik je prijavljen u sustav kao vlasnik automobila
	\item  \textbf{Opis osnovnog tijeka:}

	\item[] \begin{packed_enum}

		\item Vlasnik automobila odabire opciju upravljanja svojim automobilima
		\item Sustav prikaže sučelje s listom automobila
		\item Vlasnik automobila zatim odabere automobil i sustav prikaže podatke o tom
		automobilu, nakon što ih dohvati iz baze podataka

	\end{packed_enum}
\end{packed_item}

\noindent \underbar{\textbf{UC10 -Brisanje automobila}}
\begin{packed_item}

	\item \textbf{Glavni sudionik: } Vlasnik automobila
	\item  \textbf{Cilj:} Brisanje automobila
	\item  \textbf{Sudionici:} Baza podataka
	\item  \textbf{Preduvjet:} Korisnik je prijavljen u sustav kao vlasnik automobila
	\item  \textbf{Opis osnovnog tijeka:}

	\item[] \begin{packed_enum}

		\item Vlasnik automobila odabire opciju upravljanja svojim automobilima
		\item Sustav prikaže sučelje s listom automobila
		\item Vlasnik automobila zatim odabere automobil koji će izbrisati
		\item U bazu podataka se pohranuje promjena
		\item Sustav preusmjerava vlasnika automobila u prijašnje sučelje za upravljanje
		automobilima i prikazuje poruku za uspješno brisanje odabranog automobila

	\end{packed_enum}
	\item  \textbf{Opis mogućih odstupanja:}

	\item[] \begin{packed_item}

		\item[3.a] Vlasnik automobila je odustao od brisanja
		\item[] \begin{packed_enum}

			\item Vlasniku automobila se prikazuje izbornik hoće li izbrisati odabrani
			automobil i preusmjerava ga u sučelje ovisno o odabiru

		\end{packed_enum}

		\item[3.b] Postoji otvoren servisni nalog za automobil koji se želi
		obrisati
		\item[] \begin{packed_enum}

			\item Vlasniku automobila se prikazuje poruka o nemogućnosti brisanja odabranog
			automobila

		\end{packed_enum}

	\end{packed_item}

\end{packed_item}
\noindent \underbar{\textbf{UC11 -Otvaranje novog servisnog naloga}}
\begin{packed_item}

	\item \textbf{Glavni sudionik: } Vlasnik automobila
	\item  \textbf{Cilj:} Otvaranje novog radnog naloga za servis automobila
	\item  \textbf{Sudionici:} Baza podataka
	\item  \textbf{Preduvjet:} Korisnik je prijavljen u sustav kao vlasnik automobila
	\item  \textbf{Opis osnovnog tijeka:}

	\item[] \begin{packed_enum}

		\item Vlasnik automobila odabire opciju upravljanja servisnim nalozima
		\item Sustav prikaže sučelje s listom radnih naloga
		\item Vlasnik automobila zatim odabire opciju otvaranja novog naloga i pri
		tom odabere za koji autoservis
		\item Nakon što je stvoren radni nalog, sprema se u bazu podataka i vlasniku se prikazuje poruka za uspješno kreiranje radnog naloga te se osvježava lista radnih naloga

	\end{packed_enum}
\end{packed_item}

\noindent \underbar{\textbf{UC12 -Pregled statusa servisnog naloga}}
\begin{packed_item}

	\item \textbf{Glavni sudionik: } Vlasnik automobila
	\item  \textbf{Cilj:} Pregled statusa radnog naloga
	\item  \textbf{Sudionici:} Baza podataka
	\item  \textbf{Preduvjet:} Korisnik je prijavljen u sustav kao vlasnik automobila
	\item  \textbf{Opis osnovnog tijeka:}

	\item[] \begin{packed_enum}

		\item Vlasnik automobila odabire opciju upravljanja servisnim nalozima
		\item Sustav prikaže sučelje s listom radnih naloga dohvaćenih iz baze podataka i njihov status
		(otvoren/zatvoren/u obradi)

	\end{packed_enum}
\end{packed_item}

\noindent \underbar{\textbf{UC13 -Prihvaćanje radnog naloga}}
\begin{packed_item}

	\item \textbf{Glavni sudionik: } Serviser
	\item  \textbf{Cilj:} Prihvaćanje radnog naloga
	\item  \textbf{Sudionici:} Baza podataka
	\item  \textbf{Preduvjet:} Korisnik je prijavljen u sustav kao vlasnik autoservisa i radni nalog
	je otvoren
	\item  \textbf{Opis osnovnog tijeka:}

	\item[] \begin{packed_enum}

		\item Vlasnik autoservisa odabire opciju prihvaćanja radnog naloga
		\item Sustav prikaže listu otvorenih radnih naloga dohvaćenih iz baze podataka
		\item Vlasnik prihvaća odabrani nalog
		\item Status radnog naloga se mijenja i sprema se u bazu, te postaje dostupan svim serviserima tog autoservisa

	\end{packed_enum}
\end{packed_item}

\noindent \underbar{\textbf{UC14 -Zatvaranje radnog naloga}}
\begin{packed_item}

	\item \textbf{Glavni sudionik: } Serviser
	\item  \textbf{Cilj:} Zatvaranje radnog naloga
	\item  \textbf{Sudionici:} Baza podataka
	\item  \textbf{Preduvjet:} Korisnik je prijavljen u sustav kao serviser i radni nalog
	je prihvaćen
	\item  \textbf{Opis osnovnog tijeka:}

	\item[] \begin{packed_enum}

		\item Serviser odabire opciju rada na radnom nalogu
		\item Sustav prikaže listu dostupnih radnih naloga
		\item Serviser odabire opciju zatvaranja odabranog radnog naloga
		\item Status radnog naloga se promijeni u "zatvoren", sprema se u bazu i serviseru se prikaže poruka o uspješnom zatvaranju

	\end{packed_enum}
\end{packed_item}

\noindent \underbar{\textbf{UC15 -Uređivanje naloga redovnog servisa}}
\begin{packed_item}

	\item \textbf{Glavni sudionik: } Serviser
	\item  \textbf{Cilj:} Uređivanje prihvaćenog radnog naloga
	\item  \textbf{Sudionici:} Baza podataka
	\item  \textbf{Preduvjet:} Korisnik je prijavljen u sustav kao serviser, radni nalog je
	za redovan servis i prihvaćen
	\item  \textbf{Opis osnovnog tijeka:}

	\item[] \begin{packed_enum}

		\item Serviser odabire opciju rada na radnom nalogu
		\item Sustav prikaže listu dostupnih radnih naloga
		\item Serviser klikne na uređivanje odabranog radnog naloga za redovan
		servis
		\item U radnom nalogu serviser unosi dodatne podatke (očitana kilometraža
		vozila te preporuka za izvanredni servis ukoliko je potrebno)
		\item Promjene se zatim spremaju i pohranjuju u bazu podataka

	\end{packed_enum}
\end{packed_item}

\noindent \underbar{\textbf{UC16 -Dodavanje ugrađenog rezervnog dijela}}
\begin{packed_item}

	\item \textbf{Glavni sudionik: } Serviser
	\item  \textbf{Cilj:} Dodavanje ugrađenog rezervnog dijela u radni nalog
	\item  \textbf{Sudionici:} Baza podataka
	\item  \textbf{Preduvjet:} Korisnik je prijavljen u sustav kao serviser, radni nalog je
	za izvanredan servis i prihvaćen
	\item  \textbf{Opis osnovnog tijeka:}

	\item[] \begin{packed_enum}

		\item Serviser odabire opciju za rad na radnom nalogu izvanrednog servisa
		\item Na odabranom radnom nalogu, serviser odabire opciju dodavanja ugrađenog rezervnog dijela
		\item Serviser odabere koji će rezervni dio unijeti
		\item Promjene se zatim spremaju i pohranjuju u bazu podataka

	\end{packed_enum}
\end{packed_item}

\noindent \underbar{\textbf{UC17 -Brisanje ugrađenog rezervnog dijela}}
\begin{packed_item}

	\item \textbf{Glavni sudionik: } Serviser
	\item  \textbf{Cilj:} Izbrisati odabrani ugrađeni rezervni dio u radnom nalogu
	\item  \textbf{Sudionici:} Baza podataka
	\item  \textbf{Preduvjet:} Korisnik je prijavljen u sustav kao serviser, radni nalog je
	za izvanredan servis i prihvaćen
	\item  \textbf{Opis osnovnog tijeka:}

	\item[] \begin{packed_enum}

		\item Serviser odabire opciju rada na radnom nalogu izvanrednog servisa
		\item Na odabranom radnom nalogu, serviser odabire opciju brisanja ugrađenog rezervnog dijela
		\item Serviser odabere koji će rezervni dio izbrisati
		\item Promjene se zatim spremaju i pohranjuju u bazu podataka

	\end{packed_enum}
\end{packed_item}

\noindent \underbar{\textbf{UC18 -Brisanje usluge}}
\begin{packed_item}

	\item \textbf{Glavni sudionik: } Serviser
	\item  \textbf{Cilj:} Izbrisati uslugu
	\item  \textbf{Sudionici:} Baza podataka
	\item  \textbf{Preduvjet:} Korisnik je prijavljen u sustav kao serviser, radni nalog je
	za izvanredan servis i prihvaćen
	\item  \textbf{Opis osnovnog tijeka:}

	\item[] \begin{packed_enum}

		\item Serviser odabire opciju rada na radnom nalogu izvanrednog servisa
		\item Na odabranom radnom nalogu, serviser odabire opciju brisanja usluge
		\item Serviser odabere koju će uslugu obrisati
		\item Promjene se zatim spremaju i pohranjuju u bazu podataka

	\end{packed_enum}
\end{packed_item}

\noindent \underbar{\textbf{UC19 -Dodavanje usluge}}
\begin{packed_item}

	\item \textbf{Glavni sudionik: } Serviser
	\item  \textbf{Cilj:} Dodati uslugu
	\item  \textbf{Sudionici:} Baza podataka
	\item  \textbf{Preduvjet:} Korisnik je prijavljen u sustav kao serviser, radni nalog je
	za izvanredan servis i prihvaćen
	\item  \textbf{Opis osnovnog tijeka:}

	\item[] \begin{packed_enum}

		\item Serviser odabire opciju rada na radnom nalogu izvanrednog servisa
		\item Na odabranom radnom nalogu, serviser odabire opciju dodavanja usluge
		\item Serviser odabere uslugu za radni nalog
		\item Promjene se zatim spremaju i pohranjuju u bazu podataka

	\end{packed_enum}
\end{packed_item}

\noindent \underbar{\textbf{UC20 -Dodavanje rezervnog dijela}}
\begin{packed_item}

	\item \textbf{Glavni sudionik: } Vlasnik autoservisa
	\item  \textbf{Cilj:} Dodavanje novog rezervnog dijela za ugradnju
	\item  \textbf{Sudionici:} Baza podataka
	\item  \textbf{Preduvjet:} Korisnik je prijavljen u sustav i dodijeljeno
	mu je pravo vlasnika autoservisa
	\item  \textbf{Opis osnovnog tijeka:}

	\item[] \begin{packed_enum}

		\item Vlasnik autoservisa odabire opciju upravljanja rezervnim dijelovima
		\item Vlasnik autoservisa odabire opciju dodavanja novog rezervnog dijela
		\item Vlasnik upisuje naziv i cijenu rezervnog dijela i potvrđuje upis
		\item U bazu podataka se pohrani promjena

	\end{packed_enum}

	\item  \textbf{Opis mogućih odstupanja:}
	\item[] \begin{packed_item}
		\item[3.a] Vlasnik je unio neispravnu cijenu rezervnog dijela
		\item[] \begin{packed_enum}
			\item Vlasniku se prikazuje poruka o neispravnosti, a vlasnik je
			ispravlja i nastavlja s potvrdom
		\end{packed_enum}
	\end{packed_item}

\end{packed_item}

\noindent \underbar{\textbf{UC21 -Uređivanje rezervnog dijela}}
\begin{packed_item}

	\item \textbf{Glavni sudionik: } Vlasnik autoservisa
	\item  \textbf{Cilj:} Uređivanje postojećeg rezervnog dijela za ugradnju
	\item  \textbf{Sudionici:} Baza podataka
	\item  \textbf{Preduvjet:} Korisnik je prijavljen u sustav i dodijeljeno
	mu je pravo vlasnika autoservisa
	\item  \textbf{Opis osnovnog tijeka:}

	\item[] \begin{packed_enum}

		\item Vlasnik autoservisa odabire opciju upravljanja rezervnim dijelovima
		\item Vlasnik autoservisa iz popisa rezervnih dijelova odabere opciju
		uređivanja
		\item Vlasniku se prikazuje prozor s upisanim postojećim podacima o
		odabranom dijelu
		\item Vlasnik mijenja podatke i potvrđuje promjene
		\item U bazu podataka se pohrani promjena

	\end{packed_enum}

	\item  \textbf{Opis mogućih odstupanja:}
	\item[] \begin{packed_item}
		\item[4.a] Vlasnik je unio neispravnu cijenu rezervnog dijela
		\item[] \begin{packed_enum}
			\item Vlasniku se prikazuje poruka o neispravnosti, a vlasnik je
			ispravlja i nastavlja s potvrdom
		\end{packed_enum}
	\end{packed_item}

\end{packed_item}

\noindent \underbar{\textbf{UC22 -Brisanje rezervnog dijela}}
\begin{packed_item}

	\item \textbf{Glavni sudionik: } Vlasnik autoservisa
	\item  \textbf{Cilj:} Brisanje rezervnog dijela
	\item  \textbf{Sudionici:} Baza podataka
	\item  \textbf{Preduvjet:} Korisnik je prijavljen u sustav i dodijeljeno
	mu je pravo vlasnika autoservisa
	\item  \textbf{Opis osnovnog tijeka:}

	\item[] \begin{packed_enum}

		\item Vlasnik autoservisa odabire opciju upravljanja rezervnim dijelovima
		\item Vlasnik autoservisa iz popisa rezervnih dijelova odabere rezervni
		dio za brisanje
		\item U bazi podataka se odabrani dio označava kao nedostupan za daljnju
		ugradnju

	\end{packed_enum}
\end{packed_item}


\noindent \underbar{\textbf{UC23 -Dodavanje usluge u cjenik}}
\begin{packed_item}
	
	\item \textbf{Glavni sudionik: } Vlasnik autoservisa
	\item  \textbf{Cilj:} Dodavanje novog usluge u cjenik
	\item  \textbf{Sudionici:} Baza podataka
	\item  \textbf{Preduvjet:} Korisnik je prijavljen u sustav i dodijeljeno
	mu je pravo vlasnika autoservisa
	\item  \textbf{Opis osnovnog tijeka:}
	
	\item[] \begin{packed_enum}
		
		\item Vlasnik autoservisa odabire opciju upravljanja cjenikom usluga
		\item Vlasnik autoservisa odabire opciju dodavanja nove usluge
		\item Vlasnik upisuje naziv i cijenu usluge i potvrđuje upis
		\item U bazu podataka se pohrani promjena
		
	\end{packed_enum}
	
	\item  \textbf{Opis mogućih odstupanja:}
	\item[] \begin{packed_item}
		\item[3.a] Vlasnik je unio neispravnu cijenu usluge
		\item[] \begin{packed_enum}
			\item Vlasniku se prikazuje poruka o neispravnosti, a vlasnik je
			ispravlja i nastavlja s potvrdom
		\end{packed_enum}
	\end{packed_item}
	
\end{packed_item}

\noindent \underbar{\textbf{UC24 -Uređivanje usluge u cjeniku}}
\begin{packed_item}
	
	\item \textbf{Glavni sudionik: } Vlasnik autoservisa
	\item  \textbf{Cilj:} Uređivanje postojećeg usluge u cjeniku
	\item  \textbf{Sudionici:} Baza podataka
	\item  \textbf{Preduvjet:} Korisnik je prijavljen u sustav i dodijeljeno
	mu je pravo vlasnika autoservisa
	\item  \textbf{Opis osnovnog tijeka:}
	
	\item[] \begin{packed_enum}
		
		\item Vlasnik autoservisa odabire opciju upravljanja cjenikom usluga
		\item Vlasnik autoservisa iz popisa usluga u cjeniku odabere opciju
		uređivanja
		\item Vlasniku se prikazuje prozor s upisanim postojećim podacima o
		odabranoj usluzi
		\item Vlasnik mijenja podatke i potvrđuje promjene
		\item U bazu podataka se pohrani promjena
		
	\end{packed_enum}
	
	\item  \textbf{Opis mogućih odstupanja:}
	\item[] \begin{packed_item}
		\item[4.a] Vlasnik je unio neispravnu cijenu usluge
		\item[] \begin{packed_enum}
			\item Vlasniku se prikazuje poruka o neispravnosti, a vlasnik je
			ispravlja i nastavlja s potvrdom
		\end{packed_enum}
	\end{packed_item}
	
\end{packed_item}

\noindent \underbar{\textbf{UC25 -Brisanje usluge iz cjenika}}
\begin{packed_item}
	
	\item \textbf{Glavni sudionik: } Vlasnik autoservisa
	\item  \textbf{Cilj:} Brisanje usluge iz cjenika
	\item  \textbf{Sudionici:} Baza podataka
	\item  \textbf{Preduvjet:} Korisnik je prijavljen u sustav i dodijeljeno
	mu je pravo vlasnika autoservisa
	\item  \textbf{Opis osnovnog tijeka:}
	
	\item[] \begin{packed_enum}
		
		\item Vlasnik autoservisa odabire opciju cjenikom usluga
		\item Vlasnik autoservisa iz cjenika usluga odabire uslugu za brisanje
		\item U bazi podataka se odabrani usluga označava kao nedostupan za daljnju
		dodavanje na radne naloge
		
	\end{packed_enum}
\end{packed_item}



\noindent \underbar{\textbf{UC26 -Uređivanje podataka autoservisa}}
\begin{packed_item}

	\item \textbf{Glavni sudionik: } Vlasnik autoservisa
	\item  \textbf{Cilj:} Promjena naziva i ostalih podataka autoservisa
	\item  \textbf{Sudionici:} Baza podataka
	\item  \textbf{Preduvjet:} Korisnik je prijavljen u sustav i dodijeljeno
	mu je pravo vlasnika autoservisa
	\item  \textbf{Opis osnovnog tijeka:}

	\item[] \begin{packed_enum}

		\item Vlasnik autoservisa odabire opciju uređivanja podataka autoservisa
		\item Vlasnik autoservisa se prikazuju trenutni podaci o njegovom
		autoservisu (naziv)
		\item Vlasnik autoservisa mijenja podatke i potvrđuje promjenu
		\item Promjena se pohranjuje u bazu podataka

	\end{packed_enum}
\end{packed_item}

\noindent \underbar{\textbf{UC27 -Pregled statistike autoservisa}}
\begin{packed_item}
	
	\item \textbf{Glavni sudionik: } Vlasnik autoservisa
	\item  \textbf{Cilj:} Pregled statističkih podataka autoservisa
	\item  \textbf{Sudionici:} Baza podataka
	\item  \textbf{Preduvjet:} Korisnik je prijavljen u sustav i dodijeljeno
	mu je pravo vlasnika autoservisa
	\item  \textbf{Opis osnovnog tijeka:}
	
	\item[] \begin{packed_enum}
		
		\item Vlasnik autoservisa odabire opciju pregleda statistike
		\item Vlasnik autoservisa se prikazuju različiti statistički podaci o radu servisa (zarada, broj popravljanih automobila i ostalo)
		
	\end{packed_enum}
\end{packed_item}

\noindent \underbar{\textbf{UC28 -Dodavanje servisera}}
\begin{packed_item}

	\item \textbf{Glavni sudionik: } Vlasnik autoservisa
	\item  \textbf{Cilj:} Dodavanje novog servisera u vlastiti autoservis
	\item  \textbf{Sudionici:} Baza podataka
	\item  \textbf{Preduvjet:} Korisnik je prijavljen u sustav i dodijeljeno
	mu je pravo vlasnika autoservisa
	\item  \textbf{Opis osnovnog tijeka:}

	\item[] \begin{packed_enum}

		\item Vlasnik autoservisa odabere opciju upravljanja serviserima
		\item Vlasnik autoservisa odabere opciju dodavanja novog servisera
		\item Vlasnik autoservisa upisuje ime i prezime te korisničko ime i
		inicijalnu lozinku za korisnički račun novog servisera
		\item Novi račun se pohranjuje u bazu podataka

	\end{packed_enum}

	\item  \textbf{Opis mogućih odstupanja:}
	\item[] \begin{packed_item}
		\item[4.a] Postoji korisnički korisnički račun s istim korisničkim imenom
		\item[] \begin{packed_enum}
			\item Vlasniku se prikazuje poruka o zauzetosti korisničkog imena, a
			vlasnik bira drugo korisničko ime
		\end{packed_enum}
	\end{packed_item}

\end{packed_item}

\noindent \underbar{\textbf{UC29 -Uklanjanje servisera}}
\begin{packed_item}

	\item \textbf{Glavni sudionik: } Vlasnik autoservisa
	\item  \textbf{Cilj:} Uklanjanje servisera iz vlastitog autoservisa
	\item  \textbf{Sudionici:} Baza podataka
	\item  \textbf{Preduvjet:} Korisnik je prijavljen u sustav i dodijeljeno
	mu je pravo vlasnika autoservisa
	\item  \textbf{Opis osnovnog tijeka:}

	\item[] \begin{packed_enum}

		\item Vlasnik autoservisa odabere opciju upravljanja serviserima
		\item Vlasnik autoservisa odabere iz popisa odabire servisera kojeg želi
		ukloniti
		\item Korisnički račun servisera se briše iz baze podataka

	\end{packed_enum}
\end{packed_item}


\noindent \underbar{\textbf{UC30 -Pregled podataka
		korisničkog računa}}
\begin{packed_item}

	\item \textbf{Glavni sudionik: } Administrator
	\item  \textbf{Cilj:} Pregled podataka odabranog
	korisničkog računa
	\item  \textbf{Sudionici:} Baza podataka
	\item  \textbf{Preduvjet:} Korisnik je prijavljen u sustav i ima ulogu administratora
	\item  \textbf{Opis osnovnog tijeka:}

	\item[] \begin{packed_enum}

		\item Administrator odabere opciju upravljanja
		računima
		\item Sustav prikaže listu korisničkih računa
		\item Administrator odabere korisnički račun čije podatke želi pregledati
		\item Administratoru se prikažu korisnički podaci dohvaćeni iz baze podataka

	\end{packed_enum}
\end{packed_item}

\noindent \underbar{\textbf{UC31 -Brisanje korisničkog
		računa}}
\begin{packed_item}

	\item \textbf{Glavni sudionik: } Administrator
	\item  \textbf{Cilj:} Brisanje odabranog korisničkog računa
	\item  \textbf{Sudionici:} Baza podataka
	\item  \textbf{Preduvjet:} Korisnik je prijavljen u sustav i ima ulogu administratora

	\item  \textbf{Opis osnovnog tijeka:}

	\item[] \begin{packed_enum}

		\item Administrator odabere opciju upravljanja
		serviserima
		\item Sustav prikaže listu korisničkih računa
		\item Administrator iz popisa odabere korisnički račun kojeg želi ukloniti
		\item Korisnički račun briše se iz baze podataka

	\end{packed_enum}
\end{packed_item}

\noindent \underbar{\textbf{UC32 - Brisanje autoservisa }}
\begin{packed_item}

	\item \textbf{Glavni sudionik: } Administrator
	\item  \textbf{Cilj:} Brisanje odabranog autoservisa
	\item  \textbf{Sudionici:} Baza podataka
	\item  \textbf{Preduvjet:} Korisnik je prijavljen u sustav i ima ulogu administratora
	\item  \textbf{Opis osnovnog tijeka:}

	\item[] \begin{packed_enum}

		\item Administrator odabere opciju upravljanja autoservisima
		\item Sustav prikaže listu autoservisa
		\item Administrator iz popisa odabere autoservis kojeg želi ukloniti
		\item Autoservis se briše iz baze podataka
	\end{packed_enum}
\end{packed_item}

	\noindent \underbar{\textbf{UC33 -Pregled podataka i statistike autoservisa}}
\begin{packed_item}

	\item \textbf{Glavni sudionik: } Administrator
	\item  \textbf{Cilj:} Pregled podataka odabranog autoservisa
	\item  \textbf{Sudionici:} Baza podataka
	\item  \textbf{Preduvjet:} Korisnik je prijavljen u sustav i ima ulogu administratora
	\item  \textbf{Opis osnovnog tijeka:}

	\item[] \begin{packed_enum}

		\item Administrator odabere opciju upravljanja autoservisima
		\item Sustav prikaže listu autoservisa
		\item Administrator odabire autoservis čije podatke želi pregledati
		\item Administratoru se prikazuju osnovni i statistički podaci o autoservisu dohvaćeni iz baze podataka

	\end{packed_enum}
\end{packed_item}

%					\noindent \underbar{\textbf{UC$<$broj obrasca$>$ -$<$ime obrasca$>$}}
%					\begin{packed_item}
%
%						\item \textbf{Glavni sudionik: }$<$sudionik$>$
%						\item  \textbf{Cilj:} $<$cilj$>$
%						\item  \textbf{Sudionici:} $<$sudionici$>$
%						\item  \textbf{Preduvjet:} $<$preduvjet$>$
%						\item  \textbf{Opis osnovnog tijeka:}
%
%						\item[] \begin{packed_enum}
%
%							\item $<$opis korak jedan$>$
%							\item $<$opis korak dva$>$
%							\item $<$opis korak tri$>$
%							\item $<$opis korak četiri$>$
%							\item $<$opis korak pet$>$
%						\end{packed_enum}
%
%						\item  \textbf{Opis mogućih odstupanja:}
%
%						\item[] \begin{packed_item}
%
%							\item[2.a] $<$opis mogućeg scenarija odstupanja u koraku 2$>$
%							\item[] \begin{packed_enum}
%
%								\item $<$opis rješenja mogućeg scenarija korak 1$>$
%								\item $<$opis rješenja mogućeg scenarija korak 2$>$
%
%							\end{packed_enum}
%							\item[2.b] $<$opis mogućeg scenarija odstupanja u koraku 2$>$
%							\item[3.a] $<$opis mogućeg scenarija odstupanja  u koraku 3$>$
%
%						\end{packed_item}
%					\end{packed_item}


\subsubsection{Dijagrami obrazaca uporabe}

\begin{figure}[H]
	\includegraphics[width=\linewidth]{dijagrami/diagram0.png}
	\centering
	\caption{Dijagram obrasca uporabe, funkcionalnost posjetitelja,
		registriranog korisnika i vlasnika automobila}
	\label{fig:diagram0}
\end{figure}

\begin{figure}[H]
	\includegraphics[width=\linewidth]{dijagrami/diagram1.png}
	\centering
	\caption{Dijagram obrasca uporabe, funkcionalnost servisera i vlasnika
		autoservisa}
	\label{fig:diagram1}
\end{figure}

\begin{figure}[H]
	\includegraphics[width=\linewidth]{dijagrami/diagram2.png}
	\centering
	\caption{Dijagram obrasca uporabe, funkcionalnost administratora}
	\label{fig:diagram2}
\end{figure}


%					\textit{Prikazati odnos aktora i obrazaca uporabe odgovarajućim UML
%dijagramom. Nije nužno nacrtati sve na jednom dijagramu. Modelirati po razinama
%apstrakcije i skupovima srodnih funkcionalnosti.}
\eject

\subsection{Sekvencijski dijagrami}


\noindent \textbf{Obrazac uporabe UC1: Registracija vlasnika automobila}

\noindent Posjetitelj na početnoj stranici odabire opciju registracije. Poslužitelj posjetitelju prikazuje sučelje za registraciju. Posjetitelj zatim unosi osobne podatke i nastavlja registraciju. Sustav provjerava ispravnost podataka te ih sprema u bazu podataka ako su ispravni.Posjetitelju se na kraju prikazuje poruka o uspješnoj registraciji. Ukoliko je posjetitelj unio neispravan OIB, korisničko ime ili e-mail, ili postoji registrirani korisnik s nekim od tih podataka, posjetitelju se prikazuje poruka o neispravnosti/zauzetosti podataka.

\begin{figure}[H]
	\includegraphics[width=0.85\linewidth]{dijagrami/seq-dia-uc1.png}
	\centering
	\caption{Sekvencijski dijagram za UC1}
	\label{fig:sequence-diagram1}
\end{figure}

\eject

\noindent \textbf{Obrazac uporabe UC7: Promjena korisničkih podataka}

\noindent Korisnik odabire opciju upravljanja korisničkim računom, na što mu aplikacija dozvoljava promjenu, iz baze podataka dohvaća korisničke podatke te korisniku prikazuje sučelje za promjenu korisničkih podataka. Korisnik mijenja podatke i potom promjene potvrđuje unosom lozinke. Aplikacija provjerava točnost lozinke te ukoliko je ona ispravna i ako su novi podatci ispravni, novi podatci se spremaju u bazu podataka i korisniku se prikazuje poruka o uspješnoj promjeni podataka. Ukoliko je unesena lozinka neispravna ili ukoliko su novi podatci neispravni, korisniku se prikazuje poruka o neispravnosti unesene lozinke ili unesenih podataka.

\begin{figure}[H]
	\includegraphics[width=0.9\linewidth]{dijagrami/seq-dia-UC7.png}
	\centering
	\caption{Sekvencijski dijagram za promjenu korisničkih podataka, UC7}
	\label{fig:sequence-diagram1}
\end{figure}

\eject

\noindent \textbf{Obrasci uporabe UC12--UC15: Upravljanje radnim nalozima}
	
\noindent Vlasnik automobila (korisnik) odabire opciju upravljanja radnim nalozima, na što mu aplikacija vraća popis radnih naloga te njihov status. Tada korisnik otvara novi radni nalog koji se sprema u bazu podataka, te se uspješno spremanje javlja korisniku. Web aplikacija osvježava listu radnih naloga koja se prikazuje korisniku. Serviser javlja aplikaciji da može raditi na radnom nalogu, te se iz baze podataka tada dohvaća lista otvorenih radnih naloga, od kojih on odabire jedan. Tom radnom nalogu mijenja se status u \textit{u obradi}, te se novi status potom ažurira u bazi podataka. Serviser tijekom servisiranja upisuje kilometražu te daje preporuku za izvanredan servis. Te podatke aplikacija ažurira u bazi podataka. Nakon što serviser u aplikaciji zatvori radni nalog, status naloga se još jednom ažurira na \textit{zatvoren} u bazi podataka.
	
	\begin{figure}[H]
		\includegraphics[width=\linewidth]{dijagrami/seq-dia1.png}
		\centering
		\caption{Sekvencijski dijagram za tijek redovnog servisa, UC12--UC15}
		\label{fig:sequence-diagram1}
	\end{figure}

\eject


\noindent \textbf{Obrasci uporabe UC11-14 i UC16: Upravljanje izvanrednim servisima}

\noindent Korisnik odabire opciju upravljanja radnim nalozima, na što mu aplikacija vraća popis radnih naloga te njihov status. Tada korisnik otvara novi radni nalog koji se sprema u bazu podataka, te se uspješno spremanje javlja korisniku. Web aplikacija osvježava listu radnih naloga koja se prikazuje korisniku. Serviser zatraži listu otvorenih naloga izvanrednog servisa te prihvaća jednog od njih. Tom se radnom nalogu mijenja status u \textit{u obradi}, te se novi status potom ažurira u bazi podataka. Serviser zatim unese opis kvarova. Ako se automobilu ugradio rezervni dio serviser ih dodaje jedan po jedan i sprema promjene u bazu, a ako je obavljena neka usluga dodaje se opis popravka. Serviser ovo radi dok god još ima popravaka za obaviti nad automobilom. Promjene se ažuriraju. Nakon što serviser u aplikaciji zatvori radni nalog, status naloga se još jednom ažurira na \textit{zatvoren} u bazi podataka.

	\begin{figure}[H]
		\centering
		\includegraphics[width=\linewidth]{dijagrami/seq-dia-service2}
		\caption{Sekvencijski dijagrami tijeka izvanrednog servisa}
		\label{fig:izvanredni-dijagram}
	\end{figure}



   	
\eject

\section{Ostali zahtjevi}

			 	\begin{packed_item}

                \item Sustav treba biti implementiran u objektno-orijentiranom jeziku te omogućiti laku nadogradnju u slučaju potrebe
                \item Sustav treba biti implementiran kao web-aplikacija
                \item Korisničko sučelje mora biti responzivno te se ispravno prikazivati u preglednicima na desktop računalima i mobilnim uređajima
                \item Sučelje za vlasnike automobila treba biti jednostavno za korištenje osobama svih dobnih skupina
                \item Sučelje za servisere treba biti prilagođeno za uporabu u okruženju servisa i osigurati laku dostupnost svih funkcionalnosti
			 	\item Sustavu mogu istovremeno pristupiti više različitih vrsta korisnika
			 	\item Aplikacija mora podržavati hrvatsku abecedu pri unosu podataka
			 	\item Sustav koristi HRK kao valutu za izračun i prikaz cijena
			 	\item Pristup sustavu mora biti osiguran putem protokola HTTPS
			 	\item Podaci o automobilu se na temelju registracije trebaju provjeravati i dohvaćati iz HUO registra
			 	\item Izvršavanje dijelova aplikacije (uključujući i pristup HUO registru) mora se odviti u roku manjem od nekoliko sekundi

			 \end{packed_item}

