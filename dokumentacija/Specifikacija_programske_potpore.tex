\chapter{Specifikacija programske potpore}
		
	\section{Funkcionalni zahtjevi}
			
			\textbf{\textit{dio 1. revizije}}\\
			
			\textit{Navesti \textbf{dionike} koji imaju \textbf{interes u ovom sustavu} ili  \textbf{su nositelji odgovornosti}. To su prije svega korisnici, ali i administratori sustava, naručitelji, razvojni tim.}\\
				
			\textit{Navesti \textbf{aktore} koji izravno \textbf{koriste} ili \textbf{komuniciraju sa sustavom}. Oni mogu imati inicijatorsku ulogu, tj. započinju određene procese u sustavu ili samo sudioničku ulogu, tj. obavljaju određeni posao. Za svakog aktora navesti funkcionalne zahtjeve koji se na njega odnose.}\\
			
			
			\noindent \textbf{Dionici:}
			
			\begin{packed_enum}
				
				\item Neregistrirani korisnik
				\item Vlasnik automobila			
				\item Vlasnik autoservisa
				\item Serviser
				\item Glavni administrator
				
			\end{packed_enum}
			
			\noindent \textbf{Aktori i njihovi funkcionalni zahtjevi:}
			
			
			\begin{packed_enum}
				\item  \underbar{Neregistrirani korisnik (inicijator) može:}
				
				\begin{packed_enum}
					
					\item Registracija
					\item Uvid u servise
					\begin{packed_enum}
						
						\item  Uvid u informacije o servisima
				
					\end{packed_enum}
					
				\end{packed_enum}
			
				\item  \underbar{Vlasnik automobila (inicijator) može:}
				
				\begin{packed_enum}
					
					\item Upravljanje vozilima
					\item Pregled detalja prošlih servisa
					\item Uređivanje osobnih podataka
					\item Brisanje računa
					
				\end{packed_enum}
			
				\item  \underbar{Vlasnik autoservisa (inicijator) može:}
				
				\begin{packed_enum}
					
					\item Uređivanje cjenika usluga i podataka o autodijelovima
					\item Uvid u sve informacije o autoservisu
					\item Upravljanje listom zaposlenika
					\item Odabir je li servis redovan ili izvanredan
					\item Otvaranje i zatvaranje radnog naloga
					
				\end{packed_enum}
				
				\item  \underbar{Serviser (inicijator) može:}
				
				\begin{packed_enum}
					
					\item Registracija
					\item Nakon prijave, odabir hoće li pristupiti stranici kao serviser ili kao vlasnik automobila
					\item Unos podataka o servisu
					\item Pregled prošlih servisa
					\item Dodavanje stavki na radni nalog (pri izvanrednom servisu)
					\begin{packed_enum}
						
						\item  Dodavanje zamijenjenih rezervnih dijelova (po definiranoj cijeni)
						\item  Dodavanje rada
						
					\end{packed_enum}
					
				\end{packed_enum}
				
				\item  \underbar{Glavni administrator (inicijator) može:}
				
				\begin{packed_enum}
					
					\item Odobravanje registracije vlasnika autoservisa
					\item Upravljanje svim korisnicima
					\item Brisanje računa
					\item Uvid u statističke podatke
					
				\end{packed_enum}
			
				\item  \underbar{HUO registar (sudionik) može:}
				
				\begin{packed_enum}
					
					\item Dohvaćanje podataka o modelu i broju šasije auta na temelju registracije
					
				\end{packed_enum}
			
				\item \underbar{Baza podataka (sudionik) može:}
				
				\begin{packed_enum}
					
					\item Pohranjivanje podataka i ovlasti korisnika sustava
					\item Pohranjivanje podataka o automobilima i prošlim servisiranjima automobila
					
				\end{packed_enum}
				
			\end{packed_enum}
			
			\eject 
			
			
				
			\subsection{Obrasci uporabe}
				
				\textbf{\textit{dio 1. revizije}}
				
				\subsubsection{Opis obrazaca uporabe}
					\textit{Funkcionalne zahtjeve razraditi u obliku obrazaca uporabe. Svaki obrazac je potrebno razraditi prema donjem predlošku. Ukoliko u nekom koraku može doći do odstupanja, potrebno je to odstupanje opisati i po mogućnosti ponuditi rješenje kojim bi se tijek obrasca vratio na osnovni tijek.}\\
					

					\noindent \underbar{\textbf{UC$<$broj obrasca$>$ -$<$ime obrasca$>$}}
					\begin{packed_item}
	
						\item \textbf{Glavni sudionik: }$<$sudionik$>$
						\item  \textbf{Cilj:} $<$cilj$>$
						\item  \textbf{Sudionici:} $<$sudionici$>$
						\item  \textbf{Preduvjet:} $<$preduvjet$>$
						\item  \textbf{Opis osnovnog tijeka:}
						
						\item[] \begin{packed_enum}
	
							\item $<$opis korak jedan$>$
							\item $<$opis korak dva$>$
							\item $<$opis korak tri$>$
							\item $<$opis korak četiri$>$
							\item $<$opis korak pet$>$
						\end{packed_enum}
						
						\item  \textbf{Opis mogućih odstupanja:}
						
						\item[] \begin{packed_item}
	
							\item[2.a] $<$opis mogućeg scenarija odstupanja u koraku 2$>$
							\item[] \begin{packed_enum}
								
								\item $<$opis rješenja mogućeg scenarija korak 1$>$
								\item $<$opis rješenja mogućeg scenarija korak 2$>$
								
							\end{packed_enum}
							\item[2.b] $<$opis mogućeg scenarija odstupanja u koraku 2$>$
							\item[3.a] $<$opis mogućeg scenarija odstupanja  u koraku 3$>$
							
						\end{packed_item}
					\end{packed_item}
				
					
				\subsubsection{Dijagrami obrazaca uporabe}
					
					\textit{Prikazati odnos aktora i obrazaca uporabe odgovarajućim UML dijagramom. Nije nužno nacrtati sve na jednom dijagramu. Modelirati po razinama apstrakcije i skupovima srodnih funkcionalnosti.}
				\eject		
				
			\subsection{Sekvencijski dijagrami}
				
				\textbf{\textit{dio 1. revizije}}\\
				
				\textit{Nacrtati sekvencijske dijagrame koji modeliraju najvažnije dijelove sustava (max. 4 dijagrama). Ukoliko postoji nedoumica oko odabira, razjasniti s asistentom. Uz svaki dijagram napisati detaljni opis dijagrama.}
				\eject
	
		\section{Ostali zahtjevi}
		
			\textbf{\textit{dio 1. revizije}}\\
		 
			 	\begin{packed_item}
			 	
			 	\item  Aplikacija mora biti pregledna te mora imati laku dostupnost raznim funkcionalnostima i sadržajima
			 	\item Mogućnost istovremenog pristupa aplikaciji od strane više korisnika
			 	\item U slučaju neispravnog korištenja korisničkog sučelja sustav mora prikazati odgovarajuću poruku upozorenja
			 	\item  Aplikacija mora podržavati hrvatsku abecedu pri unosu podataka 
			 	\item Sustav treba imati pristup HUO registru
			 	\item  U slučaju prijeđenog određenog broja kilometara od poslijednjeg redovnog servisa, aplikacija treba prikazati poruku preporuke redovnog servisa nakon prijave u sustav
			 	\item Sustav koristi HRK kao valutu za izračun i prikaz cijena
			 	\item Izvršavanje dijelova aplikacije mora se odviti u razumnom roku
			 	
			 \end{packed_item}
			 
			 
			 
	